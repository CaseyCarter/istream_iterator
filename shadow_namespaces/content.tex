\begingroup
\def\hd{\begin{tabular}{lll}
          \textbf{Document Number:} & {\larger\docno}                 \\
          \textbf{Date:}            & \reldate                        \\
          \textbf{Audience:}        & Library Evolution Working Group \\
          \textbf{Author:}          & Casey Carter                    \\
          \textbf{Reply to:}        & casey@carter.net
          \end{tabular}
}
\newlength{\hdwidth}
\settowidth{\hdwidth}{\hd}
\hfill\begin{minipage}{\hdwidth}\hd\end{minipage}
\endgroup

\vspace{0.5cm}
\begin{center}
\textbf{\Huge \doctitle}
\end{center}

\vspace{0.5cm}

\chapter{Scope}

Name collisions between the Standard Library and user programs that pull the
entire standard library into their namespaces - or the global namespace - via
\tcode{using namespace std;} are a perennial problem for C++. Many members of
the community - and WG21 - believe that it is LEWG's responsibility to choose
names that don't conflict with common names in user programs to avoid collisions,
whereas many others believe that \tcode{using namespace std;} is an explicit
directive requesting such collisions.

The mechanisms in place for determining what names have the potential for
collision are straightforward: (1) we guess, or (2) someone at Big Software
Company searches their codebase for collisions. Effectively, this process
insulates code at Big Software Companies that send representatives to
meetings and code in domains in which WG21 members have worked from name
collisions. Programmers at Small Software Company or who work in
underrepresented domains are left to deal with name collisions.

Our process for changing names that we believe are likely to cause collisions is
also straightforward: we either pick an inferior name, or we move the good
name\footnote{For the purposes of this argument, assume that LEWG chooses good
names for Standard Library components. If that's too great a stretch, imagine
that occurrences of the phrase ``good name'' in this proposal instead read
``chosen name.''} into a subnamespace of \tcode{std}. Note that neither of these
choices actually avoids the possibility of collisions - an inferior name or
subnamespace could still cause collisions, after all - so much as reduces our
expectation of collisions assuming that our estimates of the collision potential
of the original and new names approximate reality.

For a recent example, consider P0631R4 ``Math Constants''~\cite{P0631R4} which
adds definitions of common mathematical constants to the standard library such
as \tcode{pi} and \tcode{e}. Many LEWG regulars were unconcerned with the
collision potential of the names of these constants - people who write using
directives are asking for collisions, after all - whereas many other WG21
members were outraged at the idea that the standard library would use names with
high collision potential and demanded that the names be changed or moved into
a subnamespace of \tcode{std}. The most commonly supported fix in the reflector
discussion was to relocate the constants into \tcode{std::math}.\footnote{This
author's guess is that \tcode{math} is nearly as common a name as \tcode{pi} and
that the suggested fix doesn't significantly reduce the potential for collision
so much as spread it more thinly across a wider domain.}

Similar examples appear regularly in the Library groups: an author chooses an
appropriate name for a library component, LEWG debates whether they agree that
the chosen name is the most appropriate in an attempt to maximize the quality of
the library, and upon discovering that the name choice is \emph{too} good,
determines how to obfuscate it to avoid collisions.


the proper
and well-known names names for these constants without locating them in a new
subnamespace of \tcode{std}.

The \libconcept{View} concept requires that its models are \libconcept{Copyable},
which excludes some useful \libconcept{Range} types -
notably including coroutine generators.
This proposal suggests relaxing the copyability requirement
to allow \libconcept{Range} types which are
\libconcept{Movable} but not \libconcept{Copyable}
to model \libconcept{View} and hence be
first-class citizens in view composition so as to
be more usable with the ranges library.

\begin{libtab2}{Tony Table}{tab:tony}{l|l}{Before}{After}
\begin{codeblock}
return view::iota(0, 42)
  | view::filter(is_even)
  | view::transform([](int i) -> auto& {
      static auto r = int_range_generator(i);
      return r; // YOLO
    })
  | view::join;
\end{codeblock} & \begin{codeblock}
return view::iota(0, 42)
  | view::filter(is_even)
  | view::transform([](int i) {
      return int_range_generator(i);
    })
  | view::join;
\end{codeblock} \\ \rowsep

\begin{codeblock}
struct fib_view {
  struct iterator {
    using difference_type = int;
    using value_type = int;

    int a = 0, b = 1;

    int operator*() const { return a; }
    iterator& operator++() {
      a = std::exchange(b, a + b);
      return *this;
    }
    void operator++(int) { ++*this; }
  };

  auto begin() const { return iterator{}; }
  auto end() const { return std::unreachable; }
};
// ...
return fib_view{} | view::take(20);
\end{codeblock} & \begin{codeblock}
generator<int> fib() {
  int a = 0, b = 1;
  while (true)
    co_yield std::exchange(a, std::exchange(b, a + b));
}
// ...
return fib() | view::take(20);
\end{codeblock} \\
\end{libtab2}

\section{Revision History}
\subsection{Revision 0}
\begin{itemize}
\item Inserted many words into an empty document.
\end{itemize}

\chapter{Problem description}

\section{background}
P0896R4 ``The One Ranges Proposal''~\cite{p0896} introduced the \libconcept{View}
concept into the \Cpp working draft. \libconcept{View} refines \libconcept{Semiregular}
which refines \libconcept{Copyable} in order that \libconcept{View}s may be used in ways
similar to \libconcept{Iterator}s which also refine \libconcept{Semiregular}.
\libconcept{Semiregular}ity combined with the constant-time complexity requirement on
\libconcept{View} and \libconcept{Iterator} operations allows them to copied freely by
range adaptors and the range adaptor closure object machinery used in
\libconcept{View} composition. Since general \libconcept{Range} types aren't required to
support such usage, they are in some sense second-class citizens in \libconcept{View}
composition.

C++11 added move-only types to the language. Parts of the Standard Library have
adapted to better support move-only types, for example, we can store them in
containers. Other parts of the library have been reluctant to do so: the
algorithms
- with the exception of the serial overload of \tcode{std::for_each} -
still require copyable function objects, for example, and we still don't have a
mechanism to type erase move-only function objects. It behooves us to consider
whether the Ranges view composition machinery added by P0896R4 should support
move-only types by admitting move-only views.

Some examples to consider:
\begin{itemize}
\item \tcode{single_view}\cxxiref{range.single.view} requires its element to model
  \libconcept{CopyConstructible}. The element is stored inside an
  \tcode{optional}-like wrapper in the \tcode{single_view} object, so the
  element must model \libconcept{CopyConstructible} in order for the
  \tcode{single_view} to model \libconcept{Semiregular}. If \libconcept{View}'s
  copyability requirement were relaxed, \tcode{single_view} could support types
  that only model \libconcept{MoveConstructible}.
\item \tcode{filter_view} and \tcode{transform_view} each hold a function object.
  Those function objects are similarly required to be \libconcept{CopyConstructible}
  only so the view can model \libconcept{Semiregular}.
\end{itemize}

\section{Why views but not function object arguments?}

We've managed to put off supporting move-only types in standard library
algorithms on the basis that there's a simple workaround: add a layer of
indirection! If you can't call
\tcode{std::ranges::find_if(myvec, move_only_predicate)}
with your move-only function object, call
\tcode{std::ranges::find_if(myvec, std::ref(move_only_predicate))}
instead. This solution typically isn't a burden since calls to algorithms
typically occur in straight-line code that already has named objects for the
range / iterator arguments. If you want to pass an rvalue move-only function
object to an algorithm, stuff it into a named object as well and use
\tcode{std::ref}.

View composition differs fundamentally from a series of algorithm calls. Instead
of a series of statements interleaved with declarations, a view composition
expression is a composition of nested function calls, possibly disguised with
the \tcode{|} operator to give it a more linear appearance, that yields a view.
There's nowhere to store intermediate results and no mechanism to name
temporaries in "the middle" of such an expression. It's necessary to
find a place to store the move-only range, pass an lvalue into view composition,
and somehow ensure that the stored range outlives the composed view. This
can be challenging when a function wants to return the result of composition of
a local move-only range with some sequence of range adaptors.

Another important difference is that \libconcept{View} is a concept that's exported
for users to utilize in their own programs, whereas the requirement that the
algorithms need copyable function objects is only words in a specification. WG21
can relax a requirement expressed in specification text any time we like simply
by changing the words - conforming implementations must ensure they handle the
relaxed requirements, but there's no possiblity of breaking user programs.
Conversely, the set of requirements expressed by a concept in the Standard can
never change without potential breakage: if we strengthen a concept, existing
callers of standard library components constrained with that concept may be
broken. If we weaken a concept, third-party library components constrained with
that concept are broken by suddenly becoming underconstrained. We have one
chance to get concepts right.

\section{Coroutines on the horizon}

Generators are the intersection of coroutines and ranges. A generator is, simply
enough, a coroutine that models the \libconcept{Range} concept. A user writes a
function whose return type is a generator type that \tcode{co_yield}s
successive elements of the range
(\href{https://wandbox.org/permlink/OjvfCc8L3pYMNdU9}{LIVE}):
\begin{codeblock}
generator<int> fib() {
  int a = 0, b = 1;
  while (true)
    co_yield std::exchange(a, std::exchange(b, a + b));
}
\end{codeblock}
which is substantially more concise than the equivalent direct implementation of
such an input range
(\href{https://wandbox.org/permlink/tlqfMsvXuoz4hSy4}{LIVE}):
\begin{codeblock}
struct fib_view {
  struct iterator {
    using difference_type = int;
    using value_type = int;

    int a = 0, b = 1;

    int operator*() const { return a; }
    iterator& operator++() {
      a = std::exchange(b, a + b);
      return *this;
    }
    void operator++(int) { ++*this; }
  };

  auto begin() const { return iterator{}; }
  auto end() const { return std::unreachable; }
};
\end{codeblock}
Note that the direct range implementation models \libconcept{View}, so it can easily
pariticipate in view composition - but the same is not true for the generator
implementation.

Coroutine frames are not copyable resources, so it's not possible to implement a
generator that can produce truly independent copies. It is possible to implement
a generator that is as \libconcept{Semiregular} as are input
iterators\footnote{With thanks to Lewis Baker.}, but only by invoking the same
spooky-action-at-distance that input iterators have in which operations
performed on an input iterator (or generator) can potentially invalidate the
copies. \libconcept{Movable}-but-not-\libconcept{Copyable} input iterators (or generators)
would avoid this issue. While it may be too late for us to solve the problem for
input iterators, it would seem a shame to repeat the design error with
generators.

\section{What about range adaptors that want to copy \libconcept{View}s?}
The range adaptors in the working draft - and the range adaptors in range-v3 -
only copy views in one circumstance: when a user calls the \tcode{base()}
member function to retrieve a copy of the underlying view being adapted. None of
the adaptors copy views in the process of their normal operation. In fact, no
one has suggested to the author a potential design for a range adaptor which
\emph{does} need to copy a view.

On the other hand, we \emph{do} have concrete examples of types
that would like to model \libconcept{View}
which are not copyable. The evidence suggests that we should relax the
copyability requirement of \libconcept{View} now, and if and when an adaptor is
discovered which does need to copy a view it can be constrained with
\tcode{View<T> \&\& Copyable<T>} along with the added semantic requirement that
copies are O(1). We could introduce a \tcode{CopyableView} concept with these
requirements when that day comes to ease using this constraint, although the
lack of any suggestion of an adaptor that needs the constraint suggests it would
be premature to standardize such a concept now.

\section{Does this mean containers can be views?}
Potentially yes, although there are some messy details. \libconcept{View} should
require O(1) destruction as well as the current requirement for O(1) copies and
moves - which lack the proposed wording below corrects - which rules out many
containers. Allocator-aware containers also notably only have O(1) move assignment when
their allocator has specific properties. This paper does not propose any changes
to allow \tcode{View} to accept container types that would otherwise be rejected
by the \tcode{enable_view} heuristic, but that is a potential avenue for future
work.

Eric Niebler has expressed concerns to the author in private communication about
such a change muddying the Ranges design, with which the author agrees: we need
to tread very carefully here.


\chapter{Proposal}
Put simply, the proposal is to remove the copyability requirement from the
\libconcept{View} concept as defined in \cxxref{range.view}:
\begin{codeblock}
template<class T>
  concept View =
    Range<T> && Semiregular<T> && enable_view<T>;
\end{codeblock}
by decomposing \tcode{Semiregular<T>} into \tcode{Copyable<T> \&\&
DefaultConstructible<T>} and then replacing \libconcept{Copyable} with
\libconcept{Movable}:
\begin{codeblock}
template<class T>
  concept View =
    Range<T> && Movable<T> && DefaultConstructible<T> && enable_view<T>;
\end{codeblock}
This allows for move-only types to model \libconcept{View} without excluding any
types which model the prior formulation of the concept.

Note that
\href{https://github.com/ericniebler/range-v3/commit/11fb1f0c6ef60a61a3eb264b5c3d0d42fc4615a2}{
a similar relaxation that only applies to single-pass (input and output) views}
has been implemented in range-v3~\cite{range-v3} since June of 2017
\href{https://github.com/ericniebler/range-v3/commit/d2bd910faa75d9016f6fb124f9de46c926c49c72}{
along with the experimental range generator implementation}.

To redress the problem that the \tcode{base()} members of the range adaptors
return copies of the underlying view, we propose that each such \tcode{base()}
member be replaced to by two overloads: a \tcode{const}-qualified overload that
requires the type of the underlying view to model \tcode{CopyConstructible}, and
a \tcode{\&\&}-qualified overload that extracts the underlying view from the
adaptor (thus leaving it invalid).


\chapter{Technical specifications}
Change \cxxref{range.req.general}/2 as follows \begin{note}There's a drive-by
edit here to require O(1) destruction: the intent has always been that
\libconcept{View}s have only constant-time operations, we apparently forgot that
destruction is an operation.\end{note}:
\begin{quote}
... The \libconcept{View} concept specifies requirements on a \libconcept{Range} type with
constant-time \changed{copy and assign}{move} operations \added{and destruction}.
\end{quote}

Change \cxxref{range.view} as follows:
\begin{quote}
\pnum
The \libconcept{View} concept specifies the requirements of a \libconcept{Range} type
that has constant time \removed{copy,} move\added{ construction},
\added{move} assignment\added{, and destruction} \removed{operators}; that is,
the cost of these operations is not proportional to
the number of elements in the \libconcept{View}.

\pnum
\begin{example}
Examples of \tcode{View}s are:

\begin{itemize}
\item A \libconcept{Range} type that wraps a pair of iterators.

\item A \libconcept{Range} type that holds its elements by \tcode{shared_ptr}
and shares ownership with all its copies.

\item A \libconcept{Range} type that generates its elements on demand.
\end{itemize}

Most containers\cxxiref{containers} are not views since
\changed{copying}{destruction of} the container \changed{copies}{destroys} the
elements, which cannot be done in constant time.
\end{example}

\indexlibrary{\idxcode{enable_view}}%
\indexlibrary{\idxcode{View}}%
\begin{itemdecl}
template<class T>
  inline constexpr bool enable_view = @\seebelow@;

template<class T>
  concept View =
    Range<T> && Semiregular<T> && enable_view<T>;
\end{itemdecl}

\begin{codeblock}
template<class T>
  concept View =
    Range<T> && @\changed{Semiregular<T>}{Movable<T> \&\& DefaultConstructible<T>}@ && enable_view<T>;
\end{codeblock}
\end{quote}

Change \cxxref{range.filter.view} as follows:
\begin{quote}
\begin{codeblock}
namespace std::ranges {
  template<InputRange V, IndirectUnaryPredicate<iterator_t<V>> Pred>
    requires View<V> && is_object_v<Pred>
  class filter_view : public view_interface<filter_view<V, Pred>> {
    [...]
    constexpr filter_view(R&& r, Pred pred);

    constexpr V base() const @\added{requires CopyConstructible<V> \{ return base_}; \added{\}}@
    @\added{constexpr V base() \&\& \{ return std::move(base_); \}}@

    constexpr iterator begin();
    [...]
\end{codeblock}
[...]
\setcounter{Paras}{2}
\begin{removedblock}
\begin{itemdecl}
constexpr V base() const;
\end{itemdecl}

\begin{itemdescr}
\pnum
\effects Equivalent to: \tcode{return base_;}
\end{itemdescr}
\end{removedblock}
\end{quote}
Change \cxxref{range.transform.view}, \cxxref{range.take.view},
\cxxref{range.common.view}, and \cxxref{range.reverse.view} similarly.

Change \cxxref{range.join.view} as follows \begin{note} This is a drive-by fix
to add \tcode{base} which was unintentionally omitted from \tcode{join_view} and
\tcode{split_view} in P0896R4.\end{note}:
\begin{quote}
\begin{codeblock}
namespace std::ranges {
  template<InputRange V>
    requires View<V> && InputRange<iter_reference_t<iterator_t<V>>> &&
             (is_reference_v<iter_reference_t<iterator_t<V>>> ||
              View<iter_value_t<iterator_t<V>>>)
  class join_view : public view_interface<join_view<V>> {
    [...]
    template<InputRange R>
      requires ViewableRange<R> && Constructible<V, all_view<R>>
    constexpr explicit join_view(R&& r);

    @\added{constexpr V base() const requires CopyConstructible<V> \{ return base_; \}}@
    @\added{constexpr V base() \&\& \{ return std::move(base_); \}}@

    constexpr auto begin() {
    [...]
\end{codeblock}
[...]
\end{quote}
Add the same definitions to \cxxref{range.split.view} before the first
definition of \tcode{begin}.

Change \cxxref{range.join.iterator}/6 as follows \begin{note} Yes, I claimed above
that ``none of the adaptors copy a view in their normal operation'' despite that
this is an occurrence of exactly that which should properly be a move. Sue me.\end{note}
\begin{note} This includes a drive-by simplification of a return type from
\tcode{decltype(auto)} to \tcode{auto\&} to clarify that \tcode{update_inner}
always returns an lvalue.\end{note}:
\begin{quote}
\begin{codeblock}
auto update_inner = [this](iter_reference_t<iterator_t<Base>> x) -> @\changed{decltype(auto)}{auto\&}@ {
  if constexpr (ref_is_glvalue) // \tcode{x} is a reference
    return (x);                 // \tcode{(x)} is an lvalue
  else
    return (parent_->inner_ = view::all(@\added{std::move(}@x@\added{)}@));
};

for (; outer_ != ranges::end(parent_->base_); ++outer_) {
  auto& inner = update_inner(*outer_);
  [...]
\end{codeblock}
\end{quote}

%% TODO: implement `move_only_view` in range-v3 and cmcstl2 for validation.
