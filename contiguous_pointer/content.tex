\begingroup
\def\hd{\begin{tabular}{lll}
          \textbf{Document Number:} & {\larger\docno}       \\
          \textbf{Date:}            & \reldate              \\
          \textbf{Audience:}        & Library Working Group \\
          \textbf{Author:}          & Casey Carter          \\
          \textbf{Reply to:}        & casey@carter.net
          \end{tabular}
}
\newlength{\hdwidth}
\settowidth{\hdwidth}{\hd}
\hfill\begin{minipage}{\hdwidth}\hd\end{minipage}
\endgroup

\vspace{0.5cm}
\begin{center}
\textbf{\Huge \doctitle}
\end{center}

\vspace{0.5cm}

\chapter{Abstract}

The support for contiguous iterators in the working draft is missing a useful
feature: a mechanism to convert a contiguous iterator into a pointer that
denotes the same object. This paper proposes that \tcode{std::to_address} be
that mechanism.

\begin{libtab2}{Tony Table}{tab:tony}{l|l}{Before}{After}
\begin{codeblock}
extern "C" int some_c_api(T* ptr, size_t size);
extern "C" int other_c_api(T* first, T* last);

template<ContiguousIterator I>
int try_useful_things(I i, size_t n) {
  // Expects: [i, n) is a valid range
  @\removed{if (n == 0) \{}@
    @\removed{// Oops - can't dereference}@
    @\removed{// past-the-end iterator}@
    @\removed{throw something;}@
  @\removed{\}}@
  return some_c_api(@\removed{addressof(*i)}@, n);
}

template<ContiguousIterator I>
int try_useful_things(I i, I j) {
  // Expects: [i, j) is a valid range
  @\removed{if (i == j) \{}@
    @\removed{// Oops - can't dereference}@
    @\removed{// past-the-end iterator}@
    @\removed{throw something;}@
  }
  return other_c_api(@\removed{addressof(*i)}@,
                     @\removed{addressof(*i)}@ + (j - i));
}
\end{codeblock} & \begin{codeblock}
extern "C" int some_c_api(T* ptr, size_t size);
extern "C" int other_c_api(T* first, T* last);

template<ContiguousIterator I>
int try_useful_things(I i, size_t n) {
  // Expects: [i, n) is a valid range
  return some_c_api(@\added{to_address(i)}@, n);
}

template<ContiguousIterator I>
int try_useful_things(I i, I j) {
  // Expects: [i, j) is a valid range
  return other_c_api(@\added{to_address(i)}@,
                     @\added{to_address(j)}@);
}
\end{codeblock} \\
\end{libtab2}

\section{Revision History}
\subsection{Revision 1}
\begin{itemize}
\item Update Tony Table: C APIs can't be overloaded, and add a bit of markup to
  make the differences clear.
\item Correct bad pointer arithmetic in the description of the address of a
  past-the-end iterator whose predecessor is dereferenceable.
\end{itemize}

\subsection{Revision 0}
\begin{itemize}
\item Initial revision.
\end{itemize}

\chapter{Problem description}

P0944R0 ``Contiguous ranges''~\cite{P0944} proposed support for contiguous
ranges and iterators, which was merged into P0896R4 ``The One Ranges
Proposal''~\cite{P0896R4} and then merged into the Working Draft. Neither
P0944R0 nor P0896R4 proposed a means of obtaining a pointer to the element
denoted by an arbitrary \tcode{ContiguousIterator}. At the time, the author was
under the impression that such a mechanism had been a ``third rail'' for past
contiguous iterator proposals~\cite{N4183}, and that requiring such a mechanism
would make it impossible to require the iterators of the Standard Library
containers to model \tcode{ContiguousIterator}. Those implementability concerns
have since been rectified.

Note that obtaining a pointer value from a dereferenceable
\tcode{ContiguousIterator} is trivial: \tcode{std::addressof(*i)} returns such a
pointer value for a contiguous iterator \tcode{i}. Dereferencing a
non-dereferenceable iterator is (unsurprisingly) not well-defined, so this
mechanism isn't suitable for iterators not known to be dereferenceable.
Obtaining a pointer value for the potentially non-dereferenceable iterator
\tcode{j} that is the past-the-end iterator of a range \range{i}{j} thus
requires a different mechanism that is well-defined for past-the-end iterators.
Ideally the mechanism would also be well-defined for dereferenceable iterators
so it can be used uniformly.

P0653R2 ``Utility to convert a pointer to a raw pointer''~\cite{P0653R2} added
the function \tcode{std::to_address}\cxxiref{pointer.conversion} to the Standard
Library which converts values of so-called ``fancy'' pointer types and standard
smart pointer types to pointer values. In the interest of spelling similar
things  similarly, it seems a good idea to reuse this facility to convert
\tcode{ContiguousIterator}s to pointer values. In practice, that means that a
type \tcode{I} must be a pointer type or
\begin{itemize}
\item specialize \tcode{pointer_traits<I>} with a member \tcode{element_type}
  or have a nested member \tcode{element_type}
  so instantiation of \tcode{pointer_traits<I>} succeeds, and
\item Either implement \tcode{pointer_traits<I>::to_address} or admit
  past-the-end (potentially non-dereferenceable) iterator values in
  \tcode{operator->()}.
\end{itemize}

\chapter{Proposal}

The basic proposal is to add a requirement to the
\libconcept{ContiguousIterator} concept that the expression
\tcode{std::to_address(i)} for an lvalue \tcode{i} of type \tcode{const I} must
\begin{itemize}
\item be well-formed and yield a pointer of type
  \tcode{add_pointer_t<iter_reference_t<i>>},
\item be well-defined for both dereferenceable and past-the-end pointer values,
\item yield a pointer value equal to \tcode{std::addressof(*i)} if \tcode{i} is
  dereferenceable, or \tcode{1 + std::addressof(*(i - 1))} if \tcode{i - 1} is
  dereferenceable.
\end{itemize}

Since dereferenceable \libconcept{ContiguousIterators} always denote objects -
their reference types are always lvalue references - they can always feasibly
implement the \tcode{->} operator. \tcode{->} is useful in contexts where the
value type of the iterator is concrete, so we propose requiring it for all
\libconcept{ContiguousIterators}.
\begin{note}
Recall that the iterator concepts do not generally require \tcode{operator->} as
do the ``old'' iterator requirements.
\end{note}

Now that there's a mechanism to retrieve a pointer from a potentially
non-dereferenceable iterator, we can also cleanup the edge cases in
\tcode{ranges::data} and \tcode{ranges::view_interface::data} which return
\tcode{nullptr} for an empty \tcode{ContiguousRange} rather than unconditionally
returning the pointer value that the \tcode{begin} iterator denotes.

\chapter{Technical specifications}
Change \cxxref{iterator.concept.contiguous} as follows:
\begin{quote}
\begin{codeblock}
template<class I>
  concept @\libconcept{ContiguousIterator}@ =
    RandomAccessIterator<I> &&
    DerivedFrom<@\placeholdernc{ITER_CONCEPT}@(I), contiguous_iterator_tag> &&
    is_lvalue_reference_v<iter_reference_t<I>> &&
    Same<iter_value_t<I>, remove_cvref_t<iter_reference_t<I>>>@\changed{;}{ \&\&}@
    @\added{requires(const I\& i) \{}@
      @\added{\{ to_address(i) \} -> Same<add_pointer_t<iter_reference_t<I>>>;}@
    @\added{\} \&\&}@
    @\added{(is_pointer_v<I> || requires(const I\& i) \{}@
      @\added{\{ i.operator->() \} -> Same<add_pointer_t<iter_reference_t<I>>>;}@
    @\added{\});}@
\end{codeblock}

\setcounter{Paras}{1}
\pnum
Let \tcode{a} and \tcode{b} be dereferenceable iterators \added{and \tcode{c} a
non-dereferenceable iterator} of type \tcode{I}
such that \tcode{b} is reachable from \tcode{a} \added{and \tcode{c} is
reachable from \tcode{b}},
and let \tcode{D} be \tcode{iter_difference_t<I>}.
The type \tcode{I} models \libconcept{ContiguousIterator} only if
\begin{removedblock}
\tcode{addressof(*(a + D(b - a)))}
is equal to
\tcode{addressof(*a) + D(b - a)}.
\end{removedblock}
\begin{addedblock}
\begin{itemize}
\item \tcode{to_address(b) == to_address(a) + D(b - a)},
\item \tcode{to_address(c) == to_address(a) + D(c - a)}, and
\item if \tcode{I} is not a pointer type, \tcode{a.operator->() == to_address(a)}.
\end{itemize}
\end{addedblock}
\end{quote}

Change \cxxref{range.prim.data} as follows:
\begin{quote}
\setcounter{Paras}{0}
\pnum
The name \tcode{data} denotes a customization point
object\cxxiref{customization.point.object}. The expression
\tcode{ranges::data(E)} for some subexpression \tcode{E} is
expression-equivalent to:

\begin{itemize}
\item
  If \tcode{E} is an lvalue, \tcode{\placeholdernc{decay-copy}(E.data())}
  if it is a valid expression of pointer to object type.

\item
  Otherwise, if \tcode{ranges::begin(E)} is a valid expression whose type models
  \tcode{ContiguousIterator}, \tcode{\added{to_address(ranges::begin(E))}}\added{.}
  \begin{removedblock}
  \begin{codeblock}
  ranges::begin(E) == ranges::end(E) ? nullptr : addressof(*ranges::begin(E))
  \end{codeblock}

  except that \tcode{E} is evaluated only once.
  \end{removedblock}

\item
  Otherwise, \tcode{ranges::data(E)} is ill-formed.
  \begin{note}
  This case can result in substitution failure when \tcode{ranges::data(E)}
  appears in the immediate context of a template instantiation.
  \end{note}
\end{itemize}
\end{quote}

Change \cxxref{view.interface} as follows:
\begin{quote}
\begin{codeblock}
namespace std::ranges {
  template<class D>
    requires is_class_v<D> && Same<D, remove_cv_t<D>>
  class view_interface : public view_base {
    [...]

    constexpr auto data() requires ContiguousIterator<iterator_t<D>> {
      @\removed{return ranges::empty(derived()) ? nullptr : addressof(*ranges::begin(derived()));}@
      @\added{return to_address(ranges::begin(derived()));}@
    }
    constexpr auto data() const
      requires Range<const D> && ContiguousIterator<iterator_t<const D>> {
        @\removed{return ranges::empty(derived()) ? nullptr : addressof(*ranges::begin(derived()));}@
        @\added{return to_address(ranges::begin(derived()));}@
      }

    [...]
  };
}
\end{codeblock}
\end{quote}
