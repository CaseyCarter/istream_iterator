\rSec0[intro]{Abstract}

This paper proposes several small, independent design tweaks to Ranges that came
up during LWG review of P0896 ``The One Ranges Proposal''~(\cite{P0896}).

All wording sections herein are relative to the post-San Diego working draft.

\rSec1[intro.history]{Revision History}
\rSec2[intro.history.r1]{Revision 2}
\begin{itemize}
\item Stuff goes here.
\end{itemize}
\rSec2[intro.history.r1]{Revision 1}
\begin{itemize}
\item Rebase wording onto post-San Diego working draft.
\item Strike section that suggested making the exposition-only concepts
  in \cxxref{special.mem.concepts} available to users; this part of the proposal
  did not have consensus in LEWG.
\item Update \tcode{safe_subrange_t} to account for potential \tcode{dangling}
  as well.
\end{itemize}
\rSec2[intro.history.r0]{Revision 0}
\begin{itemize}
\item In the beginning, all was \cv-\tcode{void}. Suddenly, a proposal emerged
  from the darkness!
\end{itemize}

\rSec0[disarm]{Deprecate \tcode{move_iterator::operator->}}

C++17 [iterator.requirements.general]/1 states:
\begin{quote}
... An iterator \tcode{i} for which the expression \tcode{(*i).m} is well-defined
supports the expression \tcode{i->m} with the same semantics as \tcode{(*i).m}. ...
\end{quote}
Input iterators are required to support the \tcode{->}
operator\cxxiref{input.iterators},
and \tcode{move_iterator} is an input iterator,
so \tcode{move_iterator}'s arrow operator must satisfy that requirement, right?
Sadly, it does not.

For a \tcode{move_iterator}, \tcode{*i} is an xvalue, so \tcode{(*i).m} is also
an xvalue. \tcode{i->m}, however, is an lvalue. Consequently, \tcode{(*i).m} and
\tcode{i->m} can produce observably different behaviors as subexpressions - they
are not substitutable, as would be expected from a strict reading of ``with the
same semantics.'' The fact that \tcode{->} cannot be implemented with ``the same
semantics'' for iterators whose reference type is an rvalue was the primary
motivation for removing the \tcode{->} requirement from the Ranges iterator
concepts. It would benefit users to deprecate \tcode{move_iterator}'s
\tcode{operator->} in C++20 as an indication that its semantics are \textit{not}
equivalent and that it will ideally go away some day.

\rSec1[disarm.words]{Technical Specifications}

\begin{itemize}
\item Strike \tcode{move_iterator::operator->} from the class template synopsis
  in [move.iterator]:
  \begin{quote}
  \begin{codeblock}
namespace std {
  template<class Iterator>
  class move_iterator {
    [...]
    constexpr iterator_type base() const;
    constexpr reference operator*() const;
    @\removed{constexpr pointer operator->() const;}@

    constexpr move_iterator& operator++();
    constexpr auto operator++(int);
    [...]
  };
}
  \end{codeblock}
  \end{quote}
\item Relocate the detailed specification of \tcode{move_iterator::operator->}
  from [move.iter.elem]:
  \begin{quote}
\begin{itemdecl}
constexpr reference operator*() const;
\end{itemdecl}
\setcounter{Paras}{0}
\begin{itemdescr}
\pnum \effects Equivalent to: \tcode{return ranges::iter_move(current);}
\end{itemdescr}
\begin{removedblock}
\begin{itemdecl}
constexpr pointer operator->() const;
\end{itemdecl}
\begin{itemdescr}
\pnum \returns \tcode{current}.
\end{itemdescr}
\end{removedblock}
\begin{itemdecl}
constexpr reference operator[](difference_type n) const;
\end{itemdecl}
\begin{itemdescr}
\pnum \effects Equivalent to: \tcode{ranges::iter_move(current + n);}
\end{itemdescr}
  \end{quote}
  to a new subclause ``Deprecated \tcode{move_iterator} access'' in Annex D
  between \cxxref{depr.iterator.primitives} and \cxxref{depr.util.smartptr.shared.atomic}:
  \begin{quote}
  \begin{addedblock}
\setcounter{Paras}{0}
\pnum
The following member is declared in addition to those members specified in
\cxxref{move.iterator.elem}:
\begin{codeblock}
namespace std {
  template<class Iterator>
  class move_iterator {
  public:
    constexpr pointer operator->() const;
  };
}
\end{codeblock}
\begin{itemdecl}
constexpr pointer operator->() const;
\end{itemdecl}
\begin{itemdescr}
\pnum \returns \tcode{current}.
\end{itemdescr}
  \end{addedblock}
  \end{quote}
\end{itemize}


\rSec0[ref]{\placeholder{ref-view} => \tcode{ref_view}}

The authors of P0896 added the exposition-only view type \placeholder{ref-view}
(P0896R4 \cxxref{range.view.ref}) to serve as the return type of  \tcode{view::all}
(\cxxref{range.adaptors.all}) when passed an lvalue container.
\tcode{\placeholder{ref-view}<T>} is an ``identity view adaptor'' --
an adaptor which produces a view containing all the elements of the underlying
range exactly -- of a \libconcept{Range} of type \tcode{T} whose representation
consists of a \tcode{T*}. A \tcode{\placeholder{ref-view}} delegates all
operations through that pointer to the underlying \libconcept{Range}.

The LEWG-approved design from
P0789R3 ``Range Adaptors and Utilities''~(\cite{P0789})
used \tcode{subrange<iterator_t<R>, sentinel_t<R>>} as the return type of
\tcode{view::all(c)} for an lvalue \tcode{c} of type \tcode{R}.
\placeholder{ref-view} and \tcode{subrange} are both identity view adaptors, so
this change has little to no impact on the existing design. Why bother then?
Despite that replacing \tcode{subrange} with \placeholder{ref-view} in this case
falls under as-if, \placeholder{ref-view} has some advantages.

Firstly, a smaller representation: \placeholder{ref-view} is a single pointer,
whereas \tcode{subrange} is an iterator plus a sentinel, and sometimes a size.
View compositions store many views produced by \tcode{view::all}, and many of
those are views of lvalue containers in typical usage.

Second, and more significantly, \placeholder{ref-view} is future-proof.
\placeholder{ref-view} retains the exact type of the underlying
\libconcept{Range}, whereas \tcode{subrange} erases down to the
\libconcept{Range}'s iterator and sentinel type. \placeholder{ref-view}
can therefore easily model any and all concepts that the underlying range models
simply by implementing any required expressions via delegating to the actual
underlying range, but \tcode{subrange} must store somewhere in its
representation any properties of the underlying range needed to model a
concept which it cannot retrieve from an iterator and sentinel. For example,
\tcode{subrange} must store a size to model \libconcept{SizedRange} when the
underlying range is sized but does not have an iterator and sentinel that model
\libconcept{SizedSentinel}. If we discover in the future that it is desirable
to have the \libconcept{View} returned by \tcode{view::all(container)} model
additional concepts, we will likely be blocked by ABI concerns with
\tcode{subrange} whereas \placeholder{ref-view} can simply add more member
functions and leave its representation unchanged.

We've already realized these advantages for view composition by adding
\placeholder{ref-view} as an exposition-only \libconcept{View} type returned
by \tcode{view::all}, but users may like to use it as well as a sort of "Ranges
\tcode{reference_wrapper}".

\rSec1[ref.words]{Technical Specifications}

\begin{itemize}
\item Update references to the name \placeholder{ref-view} to
  \tcode{ref_view} in \cxxref{range.adaptors.all}/2:
  \begin{quote}
\setcounter{Paras}{1}
\pnum
The name \tcode{view::all} denotes a range adaptor
object\cxxiref{range.adaptor.object}. For some subexpression \tcode{E},
the expression \tcode{view::all(E)} is expression-equivalent to:

\begin{itemize}
\item \tcode{\placeholdernc{DECAY_COPY}(E)} if the decayed type of \tcode{E}
models \libconcept{View}.

\item Otherwise,
\tcode{\changed{\placeholder{ref-view}\{E\}}{ref_view\{E\}}} if that
expression is well-formed\removed{, where \tcode{\placeholder{ref-view}}
is the exposition-only \libconcept{View} specified below}.

\item Otherwise, \tcode{subrange\{E\}}.
\end{itemize}

  \end{quote}
\item Change the stable name \cxxref{range.view.ref} to [range.ref.view] (for consistency with the stable names of the other view classes defined in \cxxref{range}), retitle to ``class template \tcode{ref_view}'' and modify as
  follows:
  \begin{quote}
\setcounter{Paras}{0}
\pnum
\tcode{\added{ref_view}}\added{ is a \libconcept{View} of the elements of some other \libconcept{Range}.}

\begin{codeblock}
namespace std::ranges {
  template<Range R>
    requires is_object_v<R>
  class @\changed{\placeholder{ref_view}}{ref_view}@ : public view_interface<@\changed{\placeholder{ref_view}}{ref_view}@<R>> {
  private:
    R* r_ = nullptr; // \expos
  public:
    constexpr @\changed{\placeholder{ref_view}}{ref_view}@() noexcept = default;

    template<@\placeholder{not-same-as}<\changed{\placeholder{ref-view}}{ref_view}@> T>
      requires @\seebelow@
    constexpr @\changed{\placeholder{ref_view}}{ref_view}@(T&& t);

    constexpr R& base() const;

    constexpr iterator_t<R> begin() const { return ranges::begin(*r_); }
    constexpr sentinel_t<R> end() const { return ranges::end(*r_); }

    constexpr bool empty() const
      requires requires { ranges::empty(*r_); }
    { return ranges::empty(*r_); }

    constexpr auto size() const requires SizedRange<R>
    { return ranges::size(*r_); }

    constexpr auto data() const requires ContiguousRange<R>
    { return ranges::data(*r_); }

    friend constexpr iterator_t<R> begin(@\changed{\placeholder{ref_view}}{ref_view}@ r)
    { return r.begin(); }

    friend constexpr sentinel_t<R> end(@\changed{\placeholder{ref_view}}{ref_view}@ r)
    { return r.end(); }
  };
}
\end{codeblock}

\setcounter{Paras}{0}
\begin{itemdecl}
template<@\placeholder{not-same-as}<\changed{\placeholder{ref-view}}{ref_view}@> T>
  requires @\seebelow@
constexpr @\changed{\placeholder{ref_view}}{ref_view}@(T&& t);
\end{itemdecl}

\begin{itemdescr}
[...]
\end{itemdescr}
  \end{quote}
\end{itemize}


\rSec0[untemp]{Comparison function object untemplates}

During LWG review of P0896's comparison function objects
(P0896R3 [range.comparisons]) we were asked, ``Why are we propagating the design
of the \tcode{std} comparison function objects, i.e. class templates that you
shouldn't specialize because they cannot be specialized consistently with the
\tcode{void} specializations that you actually should be using?'' For the Ranges
TS, it was a design goal to minimize differences between \tcode{std} and
\tcode{ranges} to ease transition and experimentation. For the Standard, our
goal should not be to minimize differences but to produce the best design. (As
was evidenced by the LEWG poll in Rapperswil suggesting that we should not be
afraid to diverge \tcode{std} and \tcode{ranges} components when there are reasons
to do so.)

Absent a good reason to mimic the \tcode{std} comparison function objects
exactly, we propose un-\tcode{template}-ing the \tcode{std::ranges} comparion
function objects, leaving only concrete classes with the same behavior as the
prior \tcode{void} specializations.

\rSec1[untemp.words]{Technical specifications}

In [functional.syn], modify the declarations of the \tcode{ranges}
comparison function objects as follows:
\begin{codeblock}
  [...]

  namespace ranges {
    // \cxxref{range.comparisons}, comparisons
    @\removed{template<class T = void>}@
      @\removed{requires \seebelow}@
    @\removed{struct equal_to;}@

    @\removed{template<class T = void>}@
      @\removed{requires \seebelow}@
    @\removed{struct not_equal_to;}@

    @\removed{template<class T = void>}@
      @\removed{requires \seebelow}@
    @\removed{struct greater;}@

    @\removed{template<class T = void>}@
      @\removed{requires \seebelow}@
    @\removed{struct less;}@

    @\removed{template<class T = void>}@
      @\removed{requires \seebelow}@
    @\removed{struct greater_equal;}@

    @\removed{template<class T = void>}@
      @\removed{requires \seebelow}@
    @\removed{struct less_equal;}@

    @\removed{template<>}@ struct equal_to@\removed{<void>}@;
    @\removed{template<>}@ struct not_equal_to@\removed{<void>}@;
    @\removed{template<>}@ struct greater@\removed{<void>}@;
    @\removed{template<>}@ struct less@\removed{<void>}@;
    @\removed{template<>}@ struct greater_equal@\removed{<void>}@;
    @\removed{template<>}@ struct less_equal@\removed{<void>}@;
  }

  [...]
\end{codeblock}

Update the specifications in \cxxref{range.comparisons} as well:

\setcounter{Paras}{1}
\pnum
There is an implementation-defined strict total ordering over all pointer values
of a given type. This total ordering is consistent with the partial order imposed
by the builtin operators \tcode{<}, \tcode{>}, \tcode{<=}, and \tcode{>=}.

\begin{removedblock}
\begin{itemdecl}
template<class T = void>
  requires EqualityComparable<T> || Same<T, void> || @\placeholdernc{BUILTIN_PTR_CMP}@(const T&, ==, const T&)
struct equal_to {
  constexpr bool operator()(const T& x, const T& y) const;
};
\end{itemdecl}

\begin{itemdescr}
\pnum
\tcode{operator()} has effects equivalent to:
\tcode{return ranges::equal_to<>\{\}(x, y);}
\end{itemdescr}

\begin{itemdecl}
template<class T = void>
  requires EqualityComparable<T> || Same<T, void> || @\placeholdernc{BUILTIN_PTR_CMP}@(const T&, ==, const T&)
struct not_equal_to {
  constexpr bool operator()(const T& x, const T& y) const;
};
\end{itemdecl}

\begin{itemdescr}
\pnum
\tcode{operator()} has effects equivalent to:
\tcode{return !ranges::equal_to<>\{\}(x, y);}
\end{itemdescr}

\begin{itemdecl}
template<class T = void>
  requires StrictTotallyOrdered<T> || Same<T, void> || @\placeholdernc{BUILTIN_PTR_CMP}@(const T&, <, const T&)
struct greater {
  constexpr bool operator()(const T& x, const T& y) const;
};
\end{itemdecl}

\begin{itemdescr}
\pnum
\tcode{operator()} has effects equivalent to:
\tcode{return ranges::less<>\{\}(y, x);}
\end{itemdescr}

\begin{itemdecl}
template<class T = void>
  requires StrictTotallyOrdered<T> || Same<T, void> || @\placeholdernc{BUILTIN_PTR_CMP}@(const T&, <, const T&)
struct less {
  constexpr bool operator()(const T& x, const T& y) const;
};
\end{itemdecl}

\begin{itemdescr}
\pnum
\tcode{operator()} has effects equivalent to:
\tcode{return ranges::less<>\{\}(x, y);}
\end{itemdescr}

\begin{itemdecl}
template<class T = void>
  requires StrictTotallyOrdered<T> || Same<T, void> || @\placeholdernc{BUILTIN_PTR_CMP}@(const T&, <, const T&)
struct greater_equal {
  constexpr bool operator()(const T& x, const T& y) const;
};
\end{itemdecl}

\begin{itemdescr}
\pnum
\tcode{operator()} has effects equivalent to:
\tcode{return !ranges::less<>\{\}(x, y);}
\end{itemdescr}

\begin{itemdecl}
template<class T = void>
  requires StrictTotallyOrdered<T> || Same<T, void> || @\placeholdernc{BUILTIN_PTR_CMP}@(const T&, <, const T&)
struct less_equal {
  constexpr bool operator()(const T& x, const T& y) const;
};
\end{itemdecl}

\begin{itemdescr}
\pnum
\tcode{operator()} has effects equivalent to:
\tcode{return !ranges::less<>\{\}(y, x);}
\end{itemdescr}
\end{removedblock}

\begin{itemdecl}
@\removed{template<>}@ struct equal_to@\removed{<void>}@ {
  template<class T, class U>
    requires EqualityComparableWith<T, U> || @\placeholdernc{BUILTIN_PTR_CMP}@(T, ==, U)
  constexpr bool operator()(T&& t, U&& u) const;

  using is_transparent = @\unspecnc@;
};
\end{itemdecl}

\begin{itemdescr}
\pnum
\expects
If the expression \tcode{std::forward<T>(t) == std::forward<U>(u)}
results in a call to a built-in operator \tcode{==} comparing pointers of type
\tcode{P}, the conversion sequences from both \tcode{T} and \tcode{U} to \tcode{P}
shall be equality-preserving\cxxiref{concepts.equality}.

\pnum
\effects
\begin{itemize}
\item
  If the expression \tcode{std::forward<T>(t) == std::forward<U>(u)} results in
  a call to a built-in operator \tcode{==} comparing pointers of type \tcode{P}:
  returns \tcode{false} if either (the converted value of) \tcode{t} precedes
  \tcode{u} or \tcode{u} precedes \tcode{t} in the implementation-defined strict
  total order over pointers of type \tcode{P} and otherwise \tcode{true}.

\item
  Otherwise, equivalent to:
  \tcode{return std::forward<T>(t) == std::forward<U>(u);}
\end{itemize}
\end{itemdescr}

\begin{itemdecl}
@\removed{template<>}@ struct not_equal_to@\removed{<void>}@ {
  template<class T, class U>
    requires EqualityComparableWith<T, U> || @\placeholdernc{BUILTIN_PTR_CMP}@(T, ==, U)
  constexpr bool operator()(T&& t, U&& u) const;

  using is_transparent = @\unspecnc@;
};
\end{itemdecl}

\begin{itemdescr}
\pnum
\tcode{operator()} has effects equivalent to:
\begin{codeblock}
return !ranges::equal_to@\removed{<>}@{}(std::forward<T>(t), std::forward<U>(u));
\end{codeblock}
\end{itemdescr}

\begin{itemdecl}
@\removed{template<>}@ struct greater@\removed{<void>}@ {
  template<class T, class U>
    requires StrictTotallyOrderedWith<T, U> || @\placeholdernc{BUILTIN_PTR_CMP}@(U, <, T)
  constexpr bool operator()(T&& t, U&& u) const;

  using is_transparent = @\unspecnc@;
};
\end{itemdecl}

\begin{itemdescr}
\pnum
\tcode{operator()} has effects equivalent to:
\begin{codeblock}
return ranges::less@\removed{<>}@{}(std::forward<U>(u), std::forward<T>(t));
\end{codeblock}
\end{itemdescr}

\begin{itemdecl}
@\removed{template<>}@ struct less@\removed{<void>}@ {
  template<class T, class U>
    requires StrictTotallyOrderedWith<T, U> || @\placeholdernc{BUILTIN_PTR_CMP}@(T, <, U)
  constexpr bool operator()(T&& t, U&& u) const;

  using is_transparent = @\unspecnc@;
};
\end{itemdecl}

\begin{itemdescr}
\pnum
\expects
If the expression \tcode{std::forward<T>(t) < std::forward<U>(u)} results in a
call to a built-in operator \tcode{<} comparing pointers of type \tcode{P}, the
conversion sequences from both \tcode{T} and \tcode{U} to \tcode{P} shall be
equality-preserving\cxxiref{concepts.equality}. For any expressions
\tcode{ET} and \tcode{EU} such that \tcode{decltype((ET))} is \tcode{T} and
\tcode{decltype((EU))} is \tcode{U}, exactly one of
\tcode{ranges::less\removed{<>}\{\}(ET, EU)},
\tcode{ranges::less\removed{<>}\{\}(EU, ET)}, or
\tcode{ranges::equal_to\removed{<>}\{\}(ET, EU)}
shall be \tcode{true}.

\pnum
\effects
\begin{itemize}
\item
If the expression \tcode{std::forward<T>(t) < std::forward<U>(u)} results in a
call to a built-in operator \tcode{<} comparing pointers of type \tcode{P}:
returns \tcode{true} if (the converted value of) \tcode{t} precedes \tcode{u} in
the implementation-defined strict total order over pointers of type \tcode{P}
and otherwise \tcode{false}.

\item
Otherwise, equivalent to:
\tcode{return std::forward<T>(t) < std::forward<U>(u);}
\end{itemize}
\end{itemdescr}

\begin{itemdecl}
@\removed{template<>}@ struct greater_equal@\removed{<void>}@ {
  template<class T, class U>
    requires StrictTotallyOrderedWith<T, U> || @\placeholdernc{BUILTIN_PTR_CMP}@(T, <, U)
  constexpr bool operator()(T&& t, U&& u) const;

  using is_transparent = @\unspecnc@;
};
\end{itemdecl}

\begin{itemdescr}
\pnum
\tcode{operator()} has effects equivalent to:
\begin{codeblock}
return !ranges::less@\removed{<>}@{}(std::forward<T>(t), std::forward<U>(u));
\end{codeblock}
\end{itemdescr}

\begin{itemdecl}
@\removed{template<>}@ struct less_equal@\removed{<void>}@ {
  template<class T, class U>
    requires StrictTotallyOrderedWith<T, U> || @\placeholdernc{BUILTIN_PTR_CMP}@(U, <, T)
  constexpr bool operator()(T&& t, U&& u) const;

  using is_transparent = @\unspecnc@;
};
\end{itemdecl}

\begin{itemdescr}
\pnum
\tcode{operator()} has effects equivalent to:
\begin{codeblock}
return !ranges::less@\removed{<>}@{}(std::forward<U>(u), std::forward<T>(t));
\end{codeblock}
\end{itemdescr}

Strip \tcode{<>} from occurrences of \tcode{ranges::equal_to<>},
\tcode{ranges::less<>}, etc. in: [defns.projection], [iterator.synopsis],
[commonalgoreq.general]/2, [commonalgoreq.mergeable], [commonalgoreq.sortable],
[range.syn], [range.adaptors.split_view], [algorithm.syn], [alg.find],
[alg.find.end], [alg.find.first.of], [alg.adjacent.find], [alg.count],
[alg.mismatch], [alg.equal], [alg.is_permutation], [alg.search], [alg.replace],
[alg.remove], [alg.unique], [sort], [stable.sort], [partial.sort],
[partial.sort.copy], [is.sorted], [alg.nth.element], [lower.bound],
[upper.bound], [equal.range], [binary.search], [alg.merge], [includes],
[set.union], [set.intersection], [set.difference], [set.symmetr\-ic.difference],
[push.heap], [pop.heap], [make.heap], [sort.heap], [is.heap], [alg.min.max],
[alg.lex.comparison], and [alg.permutation.generators].


\rSec0[weiv_esrever]{Reversing a \tcode{reverse_view}}

\tcode{view::reverse} in P0896 is a range adaptor that produces a
\tcode{reverse_view} which presents the elements of the underlying range
in reverse order - from back to front. \tcode{reverse_view} does so via
the expedient mechanism of adapting the underlying view's iterators with
\tcode{std::reverse_iterator}. Reversing a \tcode{reverse_view} produces a view
of the elements of the original range in their original order. While this
behavior is \tcode{correct}, it is likely to exhibit poor performance.

We propose that the effect of \tcode{view::reverse(r)} when \tcode{r} is an
instance of \tcode{reverse_view} should be to simply return the underlying view
directly. This behavior is both simple to specify and efficient to implement.

Similarly, reversing a \tcode{subrange} whose iterator and sentinel are
\tcode{reverse_iterator}s can be made more efficient by yielding a
\tcode{subrange} of "unwrapped" iterators. Note that in this case we should
take care to preserve any stored size information in the original subrange,
since the size of the unwrapped base range is the same.

\rSec1[sdrow.weiv_esrever]{Technical specifications}

\begin{itemize}
\item Modify the specification of \tcode{view::reverse} in
  \cxxref{range.reverse.adaptor} as follows:
  \begin{quote}
\pnum
The name \tcode{view::reverse} denotes a
range adaptor object\cxxiref{range.adaptor.object}.
For some subexpression \tcode{E}, the expression
\tcode{view::reverse(E)} is expression-equivalent to\added{:}
\tcode{\removed{reverse_view\{E\}}}\removed{.}
\begin{addedblock}
\begin{itemize}
\item If the type of \tcode{E} is a (possibly cv-qualified)
  specialization of \tcode{reverse_view}, \tcode{E.base()}.
\item Otherwise, if the type of \tcode{E} is cv-qualified
  \begin{codeblock}
  subrange<reverse_iterator<I>, reverse_iterator<I>, K>
  \end{codeblock}
  for some iterator type \tcode{I} and
  value \tcode{K} of \tcode{subrange_kind},
  \begin{codeblock}
  subrange<I, I, K>(E.end().base(), E.begin().base(), E.size())
  \end{codeblock}
  if \tcode{K} is \tcode{subrange_kind::sized} and
  \begin{codeblock}
  subrange<I, I, K>(E.end().base(), E.begin().base())
  \end{codeblock}
  otherwise, except that in either case \tcode{E} is evaluated only once.
\item Otherwise, \tcode{reverse_view\{E\}}.
\end{itemize}
\end{addedblock}
  \end{quote}
\end{itemize}


\rSec0[dangle]{Use cases left \tcode{dangling}}

What does this program fragment do in P0896?
\begin{codeblock}
std::vector<int> f();
o = std::ranges::copy(f(), o).out;
\end{codeblock}
how about this one:
\begin{codeblock}
std::ranges::copy(f(), std::ostream_iterator<int>{std::cout});
\end{codeblock}
The correct answer is, ``These fragments are ill-formed because the iterator into
the input range that \tcode{ranges::copy} returns would dangle - despite that
the program fragment ignores that value - because LEWG asked us to remove the
\tcode{dangling} wrapper and make such calls ill-formed.''

In the Ranges TS / revision one of P0896 an algorithm that returns an iterator
into a range that was passed as an rvalue argument first wraps that iterator
with the \tcode{dangling} wrapper template. A caller must retrieve the iterator
value from the wrapper by calling a member function, opting in to potentially
dangerous behavior explicitly. The use of \tcode{dangling} here makes it
impossible for a user to inadvertently use an iterator that dangles.

In practice, the majority of range-v3 users in an extremely rigorous poll of the
\tcode{\#ranges} Slack channel (i.e., the author and two people who responded)
never extract the value from a \tcode{dangling} wrapper. We prefer to always
pass lvalue ranges to algorithms when we plan to use the returned iterator, and
use \tcode{dangling} only as a tool to help us avoid inadvertent use of
potentially dangling iterators. Unfortunately, P0896 makes calls that would have
used \tcode{dangling} in the TS design ill-formed which forces passing ranges
as lvalues even when the dangling iterator value is not used.

We propose bringing back \tcode{dangling} in a limited capacity as a
non-template tag type to be returned by calls that would otherwise return a
dangling iterator value. This change makes the program fragments above
well-formed, but without introducing the potentially unsafe behavior that LEWG
found objectionable in the prior \tcode{dangling} design: there's no stored
iterator value to retrieve.

\rSec1[dangle.words]{Technical specifications}

Modify the \tcode{<ranges>} synopsis in \cxxref{ranges.syn} as follows:
\begin{quote}
\begin{codeblock}
namespace std::ranges {
  [...]

  // \cxxref{range.range}, Range
  template<class T>
    using iterator_t = decltype(ranges::begin(declval<T&>()));

  template<class T>
    using sentinel_t = decltype(ranges::end(declval<T&>()));

  @\removed{template<\placeholder{fowarding-range} R>}@
    @\removed{using safe_iterator_t = iterator_t<R>;}@

  template<class T>
    concept Range = @\seebelownc@;

  [...]

  template<Iterator I, Sentinel<I> S = I, subrange_kind K = @\seebelownc@>
    requires (K == subrange_kind::sized || !SizedSentinel<S, I>)
  class subrange;

  @\added{// \cxxref{dangling}, dangling iterator handling}@
  @\added{struct dangling;}@

  @\added{template<Range R>}@
    @\added{using safe_iterator_t =}@
      @\added{conditional_t<\placeholdernc{forwarding-range}<R>, iterator_t<R>, dangling>;}@

  template<@\changed{\placeholder{forwarding-range}}{Range}@ R>
    using safe_subrange_t =
      @\added{conditional_t<\placeholder{forwarding-range}<R>,}@ subrange<iterator_t<R>>@\added{, dangling>}@;

  // \cxxref{range.all}, all view
  namespace view { inline constexpr @\unspec@ all = @\unspecnc@; }

  [...]
}
\end{codeblock}
\end{quote}

Add a new subclause to \cxxref{range.utility}, following \cxxref{range.subrange}:

\begin{quote}
\begin{addedblock}
\setcounter{chapter}{23}
\setcounter{section}{6}
\setcounter{subsection}{3}
\rSec2[dangling]{Dangling iterator handling}
\pnum
The tag type \tcode{dangling} is used together with the template aliases
\tcode{safe_iterator_t} and \tcode{safe_subrange_t} to indicate that
an algorithm that typically returns an iterator into or
subrange of a \libconcept{Range} argument does not return
an iterator or subrange which could potentially dangle
for a particular rvalue \libconcept{Range} argument which
does not model \libconcept{\placeholder{forwarding-range}}\cxxiref{range.range}.

\begin{codeblock}
namespace std {
  struct dangling {
    constexpr dangling() noexcept = default;
    template<class... Args>
      constexpr dangling(Args&&...) noexcept { }
  };
}
\end{codeblock}

\pnum
\begin{example}
\begin{codeblock}
vector<int> f();
auto result1 = ranges::find(f(), 42); // \#1
static_assert(Same<decltype(result1), dangling>);
auto vec = f();
auto result2 = ranges::find(vec, 42); // \#2
static_assert(Same<decltype(result2), vector<int>::iterator>);
auto result3 = ranges::find(subrange{vec}, 42); // \#3
static_assert(Same<decltype(result3), vector<int>::iterator>);
\end{codeblock}
The call to \tcode{ranges::find} at \#1 returns \tcode{dangling} since
\tcode{f()} is an rvalue \tcode{vector}; the \tcode{vector} could potentially
be destroyed before a returned iterator is dereferenced. However, the calls
at \#2 and \#3 both return iterators since the lvalue \tcode{vec} and
specializations of \tcode{subrange} model \libconcept{forwarding-range}.
\end{example}
\end{addedblock}
\end{quote}

Change the \tcode{<algorithm>} synopsis in \cxxref{algorithm.syn} as follows:

\begin{quote}
\begin{codeblock}
#include <initializer_list>

namespace std {
  [...]
  namespace ranges {
    template<class I, class F>
    struct for_each_result {
      @\added{[[no_unique_address]]}@ I in;
      @\added{[[no_unique_address]]}@ F fun;

      @\added{template<class I2, class F2>}@
        @\added{requires ConvertibleTo<const I\&, I2> \&\& ConvertibleTo<const F\&, F2>}@
        @\added{operator for_each_result<I2, F2>() const \& \{}@
          @\added{return \{in, fun\};}@
        @\added{\}}@

      @\added{template<class I2, class F2>}@
        @\added{requires ConvertibleTo<I, I2> \&\& ConvertibleTo<F, F2>}@
        @\added{operator for_each_result<I2, F2>() \&\& \{}@
          @\added{return \{std::move(in), std::move(fun)\};}@
        @\added{\}}@
    };
    [...]
  }
  [...]
  namespace ranges {
    template<class I1, class I2>
    struct mismatch_result {
      @\added{[[no_unique_address]]}@ I1 in1;
      @\added{[[no_unique_address]]}@ I2 in2;

      @\added{template<class II1, class II2>}@
        @\added{requires ConvertibleTo<const I1\&, II1> \&\& ConvertibleTo<const I2\&, II2>}@
        @\added{operator mismatch_result<II1, II2>() const \& \{}@
          @\added{return \{in1, in2\};}@
        @\added{\}}@

      @\added{template<class II1, class II2>}@
        @\added{requires ConvertibleTo<I1, II1> \&\& ConvertibleTo<I2, II2>}@
        @\added{operator mismatch_result<II1, II2>() \&\& \{}@
          @\added{return \{std::move(in1), std::move(in2)\};}@
        @\added{\}}@
    };
    [...]
  }
  [...]
  namespace ranges {
    template<class I, class O>
    struct copy_result {
      @\added{[[no_unique_address]]}@ I in;
      @\added{[[no_unique_address]]}@ O out;

      @\added{template<class I2, class O2>}@
        @\added{requires ConvertibleTo<const I\&, I2> \&\& ConvertibleTo<const O\&, O2>}@
        @\added{operator copy_result<I2, O2>() const \& \{}@
          @\added{return \{in, out\};}@
        @\added{\}}@

      @\added{template<class I2, class O2>}@
        @\added{requires ConvertibleTo<I, I2> \&\& ConvertibleTo<O, O2>}@
        @\added{operator copy_result<I2, O2>() \&\& \{}@
          @\added{return \{std::move(in), std::move(out)\};}@
        @\added{\}}@
    };
    [...]
  }
  [...]
  namespace ranges {
    [...]
    template<class I1, class I2, class O>
    struct binary_transform_result {
      @\added{[[no_unique_address]]}@ I1 in1;
      @\added{[[no_unique_address]]}@ I2 in2;
      @\added{[[no_unique_address]]}@ O out;

      @\added{template<class II1, class II2, class OO>}@
        @\added{requires ConvertibleTo<const I1\&, II1> \&\&}@
          @\added{ConvertibleTo<const I2\&, II2> \&\& ConvertibleTo<const O\&, OO>}@
        @\added{operator binary_transform_result<II1, II2, OO>() const \& \{}@
          @\added{return \{in1, in2, out\};}@
        @\added{\}}@

      @\added{template<class II1, class II2, class OO>}@
        @\added{requires ConvertibleTo<I1, II1> \&\&}@
          @\added{ConvertibleTo<I2, II2> \&\& ConvertibleTo<O, OO>}@
        @\added{operator binary_transform_result<II1, II2, OO>() \&\& \{}@
          @\added{return \{std::move(in1), std::move(in2), std::move(out)\};}@
        @\added{\}}@
    };
    [...]
  }
  [...]
  namespace ranges {
    template<class I, class O1, class O2>
    struct partition_copy_result {
      @\added{[[no_unique_address]]}@ I in;
      @\added{[[no_unique_address]]}@ O1 out1;
      @\added{[[no_unique_address]]}@ O2 out2;

      @\added{template<class II, class OO1, class OO2>}@
        @\added{requires ConvertibleTo<const I\&, II> \&\&}@
          @\added{ConvertibleTo<const O1\&, OO1> \&\& ConvertibleTo<const O2\&, OO2>}@
        @\added{operator partition_copy_result<II, OO1, OO2>() const \& \{}@
          @\added{return \{in, out1, out2\};}@
        @\added{\}}@

      @\added{template<class II, class OO1, class OO2>}@
        @\added{requires ConvertibleTo<I, II> \&\&}@
          @\added{ConvertibleTo<O1, OO1> \&\& ConvertibleTo<O2, OO2>}@
        @\added{operator partition_copy_result<II, OO1, OO2>() \&\& \{}@
          @\added{return \{std::move(in), std::move(out1), std::move(out2)\};}@
        @\added{\}}@
    };
    [...]
  }
  [...]
  namespace ranges {
    template<class T>
    struct minmax_result {
      @\added{[[no_unique_address]]}@ T min;
      @\added{[[no_unique_address]]}@ T max;

      @\added{template<class T2>}@
        @\added{requires ConvertibleTo<const T\&, T2>}@
        @\added{operator minmax_result<T2>() const \& \{}@
          @\added{return \{min, max\};}@
        @\added{\}}@

      @\added{template<class T2>}@
        @\added{requires ConvertibleTo<T, T2>}@
        @\added{operator minmax_result<T2>() \&\& \{}@
          @\added{return \{std::move(min), std::move(max)\};}@
        @\added{\}}@
    };
    [...]
  }
  [...]
}
\end{codeblock}
\end{quote}

%% \setcounter{chapter}{7}
%% \rSec0[sub.headers]{Headers \tcode{iterator_}- and \tcode{range_concepts}}
%%
%% When defining an iterator/container/algorithm it's extremely likely that I need
%% the iterator/range concepts/traits/operations in the
%% \tcode{<iterator>}/\tcode{<ranges>} headers, but substantially less likely that I
%% need the iterator/range adaptors. Indeed, while the ``support machinery'' in
%% these headers is static, the set of useful adaptors is effectively unbounded and
%% will grow over time as more are proposed.
%%
%% In the interests of reducing the cost of transitive includes that my
%% iterator/container/algorithm header imposes on users, it would be useful if the
%% support machinery in \tcode{<iterator>}/\tcode{<ranges>} were split out into
%% separate \tcode{<iterator_support>}/\tcode{<ranges_support>}
%% (\tcode{<_base>}? \tcode{<_concepts>}?) headers which the base
%% headers include.
%%
%% \rSec1[sub.headers.words]{Technical specifications}
%%
%% Split [iterator.synopsis] between the ``range iterator operations'' and
%% ``predefined iterators and sentinels'', naming the new first subclause
%% ``Header \tcode{<iterator_support>} synopsis'' with stable name
%% [iterator.support.syn]. At what is now the beginning of the \tcode{<iterator>}
%% synopsis, add \tcode{\#include <iterator_support>}.
%%
%% Similarly split [range.syn] between the declarations of the \tcode{all_view}
%% alias and the \tcode{filter_view} view adaptor, naming the new first subclause
%% ``Header \tcode{<ranges_support>} synopsis'' with stable name
%% [range.support.syn]. At what is now the beginning of the \tcode{<ranges>}
%% synopsis, add \tcode{\#include <ranges_support>}. Relocate the declarations
%% directly in namespace \tcode{std} from \tcode{<ranges>}:
%% \begin{codeblock}
%%   namespace view = ranges::view;
%%
%%   template<class I, class S, ranges::subrange_kind K>
%%   struct tuple_size<ranges::subrange<I, S, K>>
%%     : integral_constant<size_t, 2> {};
%%   template<class I, class S, ranges::subrange_kind K>
%%   struct tuple_element<0, ranges::subrange<I, S, K>> {
%%     using type = I;
%%   };
%%   template<class I, class S, ranges::subrange_kind K>
%%   struct tuple_element<1, ranges::subrange<I, S, K>> {
%%     using type = S;
%%   };
%% \end{codeblock}
%% to the end of \tcode{<ranges_support>}.
