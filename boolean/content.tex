\begingroup
\def\hd{\begin{tabular}{lll}
          \textbf{Document Number:} & {\larger\docno}                 \\
          \textbf{Date:}            & \reldate                        \\
          \textbf{Audience:}        & Library Evolution Working Group \\
          \textbf{Authors:}         & Casey Carter                    \\
                                    & Christopher Di Bella            \\
                                    & Eric Niebler                    \\
          \textbf{Reply to:}        & casey@carter.net
          \end{tabular}
}
\newlength{\hdwidth}
\settowidth{\hdwidth}{\hd}
\hfill\begin{minipage}{\hdwidth}\hd\end{minipage}
\endgroup

\vspace{0.5cm}
\begin{center}
\textbf{\Huge \doctitle}
\end{center}

\vspace{0.5cm}

\chapter{Abstract}

The \tcode{std::boolean} concept strikes the perfect balance of being hard to validate, hard to
teach, and hard to use - as is the subject of some US NB comments. This paper proposes replacing the
failed concept with a formulation that is simple to validate, teach, and use.

\section{Revision History}
\subsection{Revision 0}
\begin{itemize}
\item Initial revision.
\end{itemize}

\chapter{Problem description}

The \libconcept{std::boolean} concept is defined in \cxxref{concept.boolean} (quoting the pre-Belfast
working draft):

\begin{quote}
\pnum
The \libconcept{boolean} concept specifies the requirements on a type that is usable as a truth
value.

\begin{codeblock}
template<class B>
  concept boolean =
    movable<remove_cvref_t<B>> &&
    requires(const remove_reference_t<B>& b1,
             const remove_reference_t<B>& b2, const bool a) {
      { b1 } -> convertible_to<bool>;
      { !b1 } -> convertible_to<bool>;
      { b1 && b2 } -> same_as<bool>;
      { b1 &&  a } -> same_as<bool>;
      {  a && b2 } -> same_as<bool>;
      { b1 || b2 } -> same_as<bool>;
      { b1 ||  a } -> same_as<bool>;
      {  a || b2 } -> same_as<bool>;
      { b1 == b2 } -> convertible_to<bool>;
      { b1 ==  a } -> convertible_to<bool>;
      {  a == b2 } -> convertible_to<bool>;
      { b1 != b2 } -> convertible_to<bool>;
      { b1 !=  a } -> convertible_to<bool>;
      {  a != b2 } -> convertible_to<bool>;
    };
\end{codeblock}

\pnum
For some type \tcode{B}, let \tcode{b1} and \tcode{b2} be lvalues of type
\tcode{const remove_reference_t<B>}. \tcode{B} models \libconcept{boolean} only if
\begin{itemize}
\item \tcode{bool(b1) == !bool(!b1)}.
\item \tcode{(b1 \&\& b2)}, \tcode{(b1 \&\& bool(b2))}, and \tcode{(bool(b1) \&\& b2)} are all equal
  to \tcode{(bool(b1) \&\& bool(b2))}, and have the same short-circuit evaluation.
\item \tcode{(b1 || b2)}, \tcode{(b1 || bool(b2))}, and \tcode{(bool(b1) || b2)} are all equal to
  \tcode{(bool(b1) || bool(b2))}, and have the same short-circuit evaluation.
\item \tcode{bool(b1 == b2)}, \tcode{bool(b1 == bool(b2))}, and \tcode{bool(bool(b1) == b2)} are all
  equal to \tcode{(bool(b1) == bool(b2))}.
\item \tcode{bool(b1 != b2)}, \tcode{bool(b1 != bool(b2))}, and \tcode{bool(bool(b1) != b2)} are all
  equal to \tcode{(bool(b1) != bool(b2))}.
\end{itemize}
\end{quote}

The Standard Library requires the results of evaluating predicate expresssions to model
\libconcept{boolean} which indicates when, for example, comparison expressions are true or false.
The concept has accreted requirements over time in an attempt to provide a more complete
approximation of the grammar and semantics of boolean expressions. It has in the process become
somewhat complex, with fourteen direct expression requirements and quite a few semantic constraints
to relate those expressions to each other. This level of definition complexity would be acceptable
if the concept is easy to apply and reason about. Unfortunately, that is not the case.

\section{Why the complexity?}
There are two different kinds of binary expressions required by \libconcept{boolean}: those that
operate on two operands of the same type (an lvalue \tcode{const B}), and those that mix a \tcode{B}
operand with a \tcode{bool} operand. Ideally, the concept would constrain how those expressions
work when applied to any pair of operands whose types each model the concept: this would achieve the
fully-general boolean grammar we want to express. There is no way to express such a universal
quantification using C++ concepts, so we fall back to applying requirements to only the type in
question and to \tcode{bool} itself.

It has been suggested that the universal quantification be applied outside of the concept syntax as
a set of purely semantic requirements. This is feasible, but perhaps not necessarily desirable. Do
we really want to have a concept that can only truly be verified by examining the entire program
text? Shouldn't I be able to reason locally, say about whether \tcode{bool} satisfies the
\libconcept{boolean} concept? In the presence of universally quantified semantic constraints a type
satisfies the concept only if there is - for example - no overloaded operator anywhere in the
program that violates the semantic requirements.

\section{Non-uniformity of application}
So the formulation of \libconcept{boolean} is necessarily a compromise; we must step outside the
required expressions when forming boolean expressions with operands of multiple types - other than
\tcode{bool} - that model the concept by converting those operands to \tcode{bool}. Consequently,
most simple uses need not force any conversions:
\begin{codeblock}
template<input_iterator I, sentinel_for<I> S, weakly_incrementable O>
  requires indirectly_copyable<I, O>
O copy(I first, S last, O out) {
  for(; first != last; ++first, (void) ++out) {
    *out = *first;
  }
  return out;
}
\end{codeblock}
which makes it easy to forget to force conversions when we venture nearer the edges of the design:
\begin{codeblock}
template<input_iterator I1, sentinel_for<I1> S1, input_iterator I2, sentinel_for<I2> S2>
  requires indirectly_comparable<I1, I2, ranges::equal>
bool equal(I1 first1, S1 last1, I2 first2, S2 last2) {
  for(; first1 != last1 && first2 != last2; // Oops; both need casts to bool
      ++first1, (void) ++first2) {
    if (*first1 != *first2)
      return false;
  }
  return first1 == last1 && first2 == last2; // Oops; here as well
}
\end{codeblock}
Style guides will presumably ``fix'' this problem by mandating that all expressions whose types
model \libconcept{boolean} are converted to \tcode{bool}, implicitly or explicitly. That simple and
universal rule makes it far easier to teach and use the concept without error, which begs the
question of what we get from \libconcept{boolean}'s current complex definition.

Certainly not ease of application and modeling. Quickly look back at the syntactic and semantic
requirements; are they satisfied by \tcode{void*}? How about \tcode{int}? Is it readily apparent that
integral types \tcode{T} model \libconcept{boolean} only over the restricted domain
\tcode{\{T(0), T(1)\}}? The authors' experience is that this formulation is far from transparent to
new users.

\subsection{Costs}
Checking fourteen \grammarterm{compound-requirement}s has a non-negligible compile time cost.
Overload resolution for each required expression and satisfaction checking of the
\grammarterm{type-requirement} are not free. Given that other concepts require often several
predicate expressions - e.g., \libconcept{sentinel_for}, \libconcept{equality_comparable}, and the
mother of all comparison concepts \libconcept{totally_ordered_with} - the checking costs are often
greatly magnified.

It's unfortunate that checking \libconcept{boolean} is so expensive, especially in the presence of
a ``always force a conversion'' style guideline which suggests that very few of the concept's required
expressions will be used in practice. The current formulation is hard to validate, hard to use, and
hard to teach.

\chapter{Proposal}

Several alternative formulations of \libconcept{boolean} have been suggested to the authors.

\begin{itemize}
\item Just use \tcode{bool}, i.e., require predicate expressions to be prvalues of type \tcode{bool}.
  This simple formulation notably doesn't need a named concept at all: we can replace uses of
  \tcode{\libconcept{boolean}<T>} (and the \grammarterm{type-requirement}
  \tcode{-> \libconcept{boolean};}) with \tcode{\libconcept{same_as}<T, bool>} (respectively
  \tcode{-> \libconcept{same_as}<bool>;}).
\item Use \tcode{\libconcept{decays_to}<T, bool>}, where \libconcept{decays_to} is:
  \begin{codeblock}
  template <class T, class U>
  concept decays_to = same_as<decay_t<T>, U>;
  \end{codeblock}
  A slight generalization of the above, this additionally allows predicate expressions that yield an
  lvalue (possibly-\tcode{const}) \tcode{bool}.
\item Use \tcode{one_of<T, bool, true_type, false_type>} where \libconcept{one_of} is:
  \begin{codeblock}
  template <class T, class... Us>
  concept one_of = (same_as<decay_t<T>, Us> && ...);
  \end{codeblock}
  This formulation cleverly covers the types for which \libconcept{boolean} support is most often
  requested.
\end{itemize}
These formulations notably never\footnote{Ignoring for the moment that a user could e.g. define
an operator overload that takes \tcode{bool} and \tcode{true_type} in the global namespace.} require
a user to force a conversion to \tcode{bool} when forming binary expressions, which is a nicely
consistent rule for usage.

The other consistent rule is ``always force a conversion,'' and there are at least two suggestions
that use that rule:
\begin{itemize}
\item ``Contextually convertible to \tcode{bool}.'' There's no convenient short spelling for
  contextual conversion to \tcode{bool}, so we'd need to keep a named concept:
  \begin{codeblock}
  template <class T>
  concept boolean = requires(const remove_reference_t<T>& t) {
    t ? 0 : 0; // The conditional operator contextually converts its first operand to bool.
  };
  \end{codeblock}
  This is the rule the [algorithms] have used for predicates for decades.
\item ``explicitly and implicitly convertible to \tcode{bool}'' which is similar to the above -
  contextual conversion to \tcode{bool} can trigger explicit conversions - with the additional
  requirement for implicit conversion. This has the short spellings
  \tcode{\libconcept{convertible_to<T, bool>}} and \tcode{-> \libconcept{convertible_to}<bool>;}.
  While this formulation is slightly more restrictive than ``contextually convertible,'' the direct
  reuse of the \libconcept{convertible_to} concept is attractive.
\end{itemize}

The authors propose this last formulation be used to replace the \libconcept{boolean} concept for
C++20. \tcode{\libconcept{convertible_to}<bool>} seems to strike a balance between the concerns
favored by the other formulations. There's a short spelling, allowing the named concept to be
removed. It admits a variety of models including all the examples currently listed as
satisfying \libconcept{boolean}. It's relatively cheap to check (as are all of the noted
alternatives). It's quite close to the ``old'' notion of what a predicate should yield. It uses the
other library concepts well: \tcode{\libconcept{convertible_to}<T>} \emph{should} be what we teach
programmers to reach for when they want to capture the set of types that convert to \tcode{T}.

\chapter{Technical Specifications} %% FIXME
Change \cxxref{cmp.concept} as follows:
\begin{quote}
\begin{codeblock}
template<class T, class U>
concept @\placeholder{partially-ordered-with}@ = // \expos
  requires(const remove_reference_t<T>& t, const remove_reference_t<U>& u) {
    { t <  u } -> @\changed{boolean}{convertible_to<bool>}@;
    { t >  u } -> @\changed{boolean}{convertible_to<bool>}@;
    { t <= u } -> @\changed{boolean}{convertible_to<bool>}@;
    { t >= u } -> @\changed{boolean}{convertible_to<bool>}@;
    { u <  t } -> @\changed{boolean}{convertible_to<bool>}@;
    { u >  t } -> @\changed{boolean}{convertible_to<bool>}@;
    { u <= t } -> @\changed{boolean}{convertible_to<bool>}@;
    { u >= t } -> @\changed{boolean}{convertible_to<bool>}@;
  };
\end{codeblock}
\end{quote}

Change \cxxref{concepts.syn} as follows:
\begin{quote}
\begin{codeblock}
namespace std {
  [...]
  // 18.5, comparison concepts
  @\removed{// 18.5.2, concept boolean}@
  @\removed{template<class B>}@
    @\removed{concept boolean = \seebelow;}@
  [...]
}
\end{codeblock}
\end{quote}

Strike the subclause \cxxref{concept.boolean}.

Change \cxxref{concept.equalitycomparable} as follows:
\begin{quote}
\begin{codeblock}
template<class T, class U>
  concept weakly-equality-comparable-with = // exposition only
    requires(const remove_reference_t<T>& t,
             const remove_reference_t<U>& u) {
      { t == u } -> @\changed{boolean}{convertible_to<bool>}@;
      { t != u } -> @\changed{boolean}{convertible_to<bool>}@;
      { u == t } -> @\changed{boolean}{convertible_to<bool>}@;
      { u != t } -> @\changed{boolean}{convertible_to<bool>}@;
    };
\end{codeblock}
[...]
\end{quote}

Change \cxxref{concept.totallyordered} as follows:
\begin{quote}
\begin{codeblock}
template<class T>
  concept totally_ordered =
    equality_comparable<T> &&
    requires(const remove_reference_t<T>& a,
             const remove_reference_t<T>& b) {
      { a <  b } -> @\changed{boolean}{convertible_to<bool>}@;
      { a >  b } -> @\changed{boolean}{convertible_to<bool>}@;
      { a <= b } -> @\changed{boolean}{convertible_to<bool>}@;
      { a >= b } -> @\changed{boolean}{convertible_to<bool>}@;
    };
\end{codeblock}
[...]
\begin{codeblock}
template<class T, class U>
  concept totally_ordered_with =
    totally_ordered<T> && totally_ordered<U> &&
    common_reference_with<const remove_reference_t<T>&, const remove_reference_t<U>&> &&
    totally_ordered<
      common_reference_t<
        const remove_reference_t<T>&,
        const remove_reference_t<U>&>> &&
    equality_comparable_with<T, U> &&
    requires(const remove_reference_t<T>& t,
             const remove_reference_t<U>& u) {
      { t <  u } -> @\changed{boolean}{convertible_to<bool>}@;
      { t >  u } -> @\changed{boolean}{convertible_to<bool>}@;
      { t <= u } -> @\changed{boolean}{convertible_to<bool>}@;
      { t >= u } -> @\changed{boolean}{convertible_to<bool>}@;
      { u <  t } -> @\changed{boolean}{convertible_to<bool>}@;
      { u >  t } -> @\changed{boolean}{convertible_to<bool>}@;
      { u <= t } -> @\changed{boolean}{convertible_to<bool>}@;
      { u >= t } -> @\changed{boolean}{convertible_to<bool>}@;
    };
\end{codeblock}
[...]
\end{quote}

Change \cxxref{concept.predicate} as follows:
\begin{quote}
\begin{codeblock}
template<class F, class... Args>
  concept predicate =
    regular_invocable<F, Args...> &&
    @\changed{boolean}{convertible_to}@<invoke_result_t<F, Args...>@\added{, bool}@>;
\end{codeblock}
\end{quote}
