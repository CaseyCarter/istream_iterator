\begingroup
\def\hd{\begin{tabular}{lll}
          \textbf{Document Number:} & {\larger\docno}                 \\
          \textbf{Date:}            & \reldate                        \\
          \textbf{Audience:}        & Library Evolution Working Group \\
          \textbf{Author:}          & Casey Carter                    \\
          \textbf{Reply to:}        & casey@carter.net
          \end{tabular}
}
\newlength{\hdwidth}
\settowidth{\hdwidth}{\hd}
\hfill\begin{minipage}{\hdwidth}\hd\end{minipage}
\endgroup

\vspace{0.5cm}
\begin{center}
\textbf{\Huge \doctitle}
\end{center}

\vspace{0.5cm}

\chapter{Abstract}

The support for contiguous iterators in the working draft is missing a useful
feature: a mechanism to convert a contiguous iterator into a pointer that
denotes the same object. This paper proposes that \tcode{std::to_address} be
that mechanism.

\begin{libtab2}{Tony Table}{tab:tony}{l|l}{Before}{After}
\begin{codeblock}
extern "C" int some_c_api(T* ptr, size_t size);
int try_useful_things(ContiguousIterator auto i, size_t n) {
  // Expects: [i, n) is a valid range
  if (n == 0) {
    // Oops - can't dereference past-the-end iterator
    throw something;
  }
  return some_c_api(addressof(*i), n);
}
\end{codeblock} & \begin{codeblock}
extern "C" int some_c_api(T* ptr, size_t size);
int try_useful_things(ContiguousIterator auto i, size_t n) {
  // Expects: [i, n) is a valid range
  return some_c_api(to_address(i), n);
}
\end{codeblock} \\ \rowsep

\begin{codeblock} %% FIXME
\end{codeblock} & \begin{codeblock}
\end{codeblock} \\
\end{libtab2}

\section{Revision History}
\subsection{Revision 0}
\begin{itemize}
\item Initial revision.
\end{itemize}

\chapter{Problem description}

P0944R0 ``Contiguous ranges''~\cite{P0944} proposed support for contiguous
ranges and iterators, which was merged into P0896R4 ``The One Ranges
Proposal''~\cite{P0896R4} and then merged into the Working Draft. Neither
P0944R0 nor P0896R4 proposed a means of obtaining a pointer to the element
denoted by an arbitrary \tcode{ContiguousIterator}. At the time, the author was
under the impression that such a mechanism had been a ``third rail'' for past
contiguous iterator proposals~\cite{N4183}, and that requiring such a mechanism
would make it impossible to require the iterators of the Standard Library
containers to model \tcode{ContiguousIterator}. Those implementability concerns
have since been rectified.

Note that obtaining a pointer value from a dereferenceable
\tcode{ContiguousIterator} is trivial: \tcode{std::addressof(*i)} returns such a
pointer value for a contiguous iterator \tcode{i}. Dereferencing a
non-dereferenceable iterator is (unsurprisingly) not well-defined, so this
mechanism isn't suitable for iterators not known to be dereferenceable.
Obtaining a pointer value for the potentially non-dereferenceable iterator
\tcode{j} that is the past-the-end iterator of a range \range{i}{j} thus
requires a different mechanism that is well-defined for past-the-end iterators.
Ideally the mechanism would also be well-defined for dereferenceable iterators
so it can be used uniformly.

P0653R2 ``Utility to convert a pointer to a raw pointer''~\cite{P0653R2} added
the function \tcode{std::to_address}\cxxiref{pointer.conversion} to the Standard
Library which converts values of so-called ``fancy'' pointer types and standard
smart pointer types to pointer values. In the
interest of spelling similar things similarly, it seems a good idea to reuse
this facility to convert \tcode{ContiguousIterator}s to pointer values.
In practice, that means that a type \tcode{I} must be a pointer type or
\begin{itemize}
\item specialize \tcode{pointer_traits<I>} with a member \tcode{element_type}
  or have a nested member \tcode{element_type}
  so instantiation of \tcode{pointer_traits<I>} succeeds, and
\item Either implement \tcode{pointer_traits<I>::to_address} or

\chapter{Proposal}

The basic proposal is to add a requirement to the
\libconcept{ContiguousIterator} concept that the expression
\tcode{std::to_address(i)} for an lvalue \tcode{i} of type \tcode{const I} must
\begin{itemize}
\item be well-formed and yield a pointer of type
  \tcode{add_pointer_t<iter_reference_t<i>>},
\item be well-defined for both dereferenceable and past-the-end pointer values,
\item yield a pointer value equal to \tcode{std::addressof(*i)} if \tcode{i} is
  dereferenceable, or equal to \tcode{std::addressof(*(i - 1))} if \tcode{i - 1}
  is dereferenceable.
\end{itemize}


\chapter{Technical specifications}
