% Definitions of table environments

%%--------------------------------------------------
%% Table environments

% Set parameters for floating tables
\setcounter{totalnumber}{10}

% Base definitions for tables
\newenvironment{TableBase}
{
 \renewcommand{\tcode}[1]{\CodeStylex{##1}}
 \newcommand{\topline}{\hline}
 \newcommand{\capsep}{\hline\hline}
 \newcommand{\rowsep}{\hline}
 \newcommand{\bottomline}{\hline}

%% vertical alignment
 \newcommand{\rb}[1]{\raisebox{1.5ex}[0pt]{##1}}	% move argument up half a row

%% header helpers
 \newcommand{\hdstyle}[1]{\textbf{##1}}				% set header style
 \newcommand{\Head}[3]{\multicolumn{##1}{##2}{\hdstyle{##3}}}	% add title spanning multiple columns
 \newcommand{\lhdrx}[2]{\Head{##1}{|c}{##2}}		% set header for left column spanning #1 columns
 \newcommand{\chdrx}[2]{\Head{##1}{c}{##2}}			% set header for center column spanning #1 columns
 \newcommand{\rhdrx}[2]{\Head{##1}{c|}{##2}}		% set header for right column spanning #1 columns
 \newcommand{\ohdrx}[2]{\Head{##1}{|c|}{##2}}		% set header for only column spanning #1 columns
 \newcommand{\lhdr}[1]{\lhdrx{1}{##1}}				% set header for single left column
 \newcommand{\chdr}[1]{\chdrx{1}{##1}}				% set header for single center column
 \newcommand{\rhdr}[1]{\rhdrx{1}{##1}}				% set header for single right column
 \newcommand{\ohdr}[1]{\ohdrx{1}{##1}}
 \newcommand{\br}{\hfill\break}						% force newline within table entry

%% column styles
 \newcolumntype{x}[1]{>{\raggedright\let\\=\tabularnewline}p{##1}}	% word-wrapped ragged-right
 																	% column, width specified by #1
 % \newcolumntype{m}[1]{>{\CodeStyle}l{##1}}              % variable width column, all entries in CodeStyle
 \newcolumntype{m}[1]{l{##1}}							% variable width column, all entries in CodeStyle

  % do not number bullets within tables
  \renewcommand{\labelitemi}{---}
  \renewcommand{\labelitemii}{---}
  \renewcommand{\labelitemiii}{---}
  \renewcommand{\labelitemiv}{---}
}
{
}

% General Usage: TITLE is the title of the table, XREF is the
% cross-reference for the table. LAYOUT is a sequence of column
% type specifiers (e.g. cp{1.0}c), without '|' for the left edge
% or right edge.

% usage: \begin{floattablebase}{TITLE}{XREF}{COLUMNS}{PLACEMENT}
% produces floating table, location determined within limits
% by LaTeX.
\newenvironment{floattablebase}[4]
{
 \begin{TableBase}
 \begin{table}[#4]
 \caption{\label{#2}#1}
 \begin{center}
 \begin{tabular}{|#3|}
}
{
 \bottomline
 \end{tabular}
 \end{center}
 \end{table}
 \end{TableBase}
}

% usage: \begin{floattable}{TITLE}{XREF}{COLUMNS}
% produces floating table, location determined within limits
% by LaTeX.
\newenvironment{floattable}[3]
{
 \begin{floattablebase}{#1}{#2}{#3}{htbp}
}
{
 \end{floattablebase}
}

% a column in a multicolfloattable (internal)
\newenvironment{mcftcol}{%
 \renewcommand{\columnbreak}{%
  \end{mcftcol} &
  \begin{mcftcol}
 }%
 \setlength{\tabcolsep}{0pt}%
 \begin{tabular}[t]{l}
}{
 \end{tabular}
}

% usage: \begin{multicolfloattable}{TITLE}{XREF}{COLUMNS}
% produces floating table, location determined within limits
% by LaTeX.
\newenvironment{multicolfloattable}[3]
{
 \begin{floattable}{#1}{#2}{#3}
 \topline
 \begin{mcftcol}
}
{
 \end{mcftcol} \\
 \end{floattable}
}

% usage: \begin{tokentable}{TITLE}{XREF}{HDR1}{HDR2}
% produces six-column table used for lists of replacement tokens;
% the columns are in pairs -- left-hand column has header HDR1,
% right hand column has header HDR2; pairs of columns are separated
% by vertical lines. Used in the "Alternative tokens" table.
\newenvironment{tokentable}[4]
{
 \begin{floattablebase}{#1}{#2}{cc|cc|cc}{htbp}
 \topline
 \hdstyle{#3}   &   \hdstyle{#4}    &
 \hdstyle{#3}   &   \hdstyle{#4}    &
 \hdstyle{#3}   &   \hdstyle{#4}    \\ \capsep
}
{
 \end{floattablebase}
}

% usage: \begin{libsumtabbase}{TITLE}{XREF}{HDR1}{HDR2}
% produces three-column table with column headers HDR1 and HDR2.
% Used in "Library Categories" table in standard, and used as
% base for other library summary tables.
\newenvironment{libsumtabbase}[4]
{
 \begin{floattable}{#1}{#2}{lll}
 \topline
 \lhdrx{2}{#3}	&	\hdstyle{#4}	\\ \capsep
}
{
 \end{floattable}
}

% usage: \begin{libsumtab}{TITLE}{XREF}
% produces three-column table with column headers "Subclause" and "Header(s)".
% Used in "C++ Headers for Freestanding Implementations" table in standard.
\newenvironment{libsumtab}[2]
{
 \begin{libsumtabbase}{#1}{#2}{Subclause}{Header(s)}
}
{
 \end{libsumtabbase}
}

% usage: \begin{concepttable}{TITLE}{TAG}{LAYOUT}
% produces table at current location
\newenvironment{concepttable}[3]
{
 \begin{TableBase}
 \begin{table}[!htb]
 \caption[#1]{\label{tab:#2}#1}
 \begin{center}
 \begin{tabular}{|#3|}
}
{
 \bottomline
 \end{tabular}
 \end{center}
 \end{table}
 \end{TableBase}
}

% usage: \begin{simpletypetable}{TITLE}{TAG}{LAYOUT}
% produces table at current location
\newenvironment{simpletypetable}[3]
{
 \begin{TableBase}
 \begin{table}[!htb]
 \caption{#1}\label{#2}
 \begin{center}
 \begin{tabular}{|#3|}
}
{
 \bottomline
 \end{tabular}
 \end{center}
 \end{table}
 \end{TableBase}
}

% usage: \begin{LongTable}{TITLE}{XREF}{LAYOUT}
% produces table that handles page breaks sensibly.
\newenvironment{LongTable}[3]
{
 \newcommand{\continuedcaption}{\caption[]{#1 (continued)}}
 \begin{TableBase}
 \begin{longtable}
 {|#3|}\caption{#1}\label{#2}
}
{
 \bottomline
 \end{longtable}
 \end{TableBase}
}

% usage: \begin{libreqtabN}{TITLE}{XREF}
% produces an N-column breakable table. Used in
% most of the library Clauses for requirements tables.
% Example at "Position type requirements" in the standard.

\newenvironment{libreqtab2}[2]
{
 \begin{LongTable}
 {#1}{#2}
 {lx{.55\hsize}}
}
{
 \end{LongTable}
}

\newenvironment{libreqtab2a}[2]
{
 \begin{LongTable}
 {#1}{#2}
 {x{.30\hsize}x{.64\hsize}}
}
{
 \end{LongTable}
}

\newenvironment{libreqtab3}[2]
{
 \begin{LongTable}
 {#1}{#2}
 {x{.28\hsize}x{.18\hsize}x{.43\hsize}}
}
{
 \end{LongTable}
}

\newenvironment{libreqtab3a}[2]
{
 \begin{LongTable}
 {#1}{#2}
 {x{.28\hsize}x{.33\hsize}x{.29\hsize}}
}
{
 \end{LongTable}
}

\newenvironment{libreqtab3b}[2]
{
 \begin{LongTable}
 {#1}{#2}
 {x{.40\hsize}x{.25\hsize}x{.25\hsize}}
}
{
 \end{LongTable}
}

\newenvironment{libreqtab3e}[2]
{
 \begin{LongTable}
 {#1}{#2}
 {x{.38\hsize}x{.27\hsize}x{.25\hsize}}
}
{
 \end{LongTable}
}

\newenvironment{libreqtab3f}[2]
{
 \begin{LongTable}
 {#1}{#2}
 {x{.35\hsize}x{.28\hsize}x{.29\hsize}}
}
{
 \end{LongTable}
}

\newenvironment{libreqtab4a}[2]
{
 \begin{LongTable}
 {#1}{#2}
 {x{.14\hsize}x{.30\hsize}x{.30\hsize}x{.14\hsize}}
}
{
 \end{LongTable}
}

\newenvironment{libreqtab4b}[2]
{
 \begin{LongTable}
 {#1}{#2}
 {x{.13\hsize}x{.15\hsize}x{.29\hsize}x{.27\hsize}}
}
{
 \end{LongTable}
}

\newenvironment{libreqtab4c}[2]
{
 \begin{LongTable}
 {#1}{#2}
 {x{.16\hsize}x{.21\hsize}x{.21\hsize}x{.30\hsize}}
}
{
 \end{LongTable}
}

\newenvironment{libreqtab4d}[2]
{
 \begin{LongTable}
 {#1}{#2}
 {x{.22\hsize}x{.22\hsize}x{.30\hsize}x{.15\hsize}}
}
{
 \end{LongTable}
}

\newenvironment{libreqtab5}[2]
{
 \begin{LongTable}
 {#1}{#2}
 {x{.14\hsize}x{.14\hsize}x{.20\hsize}x{.20\hsize}x{.14\hsize}}
}
{
 \end{LongTable}
}

% usage: \begin{libtab2}{TITLE}{XREF}{LAYOUT}{HDR1}{HDR2}
% produces two-column table with column headers HDR1 and HDR2.
% Used in "seekoff positioning" in the standard.
\newenvironment{libtab2}[5]
{
 \begin{floattable}
 {#1}{#2}{#3}
 \topline
 \lhdr{#4}	&	\rhdr{#5}	\\ \capsep
}
{
 \end{floattable}
}

% usage: \begin{longlibtab2}{TITLE}{XREF}{LAYOUT}{HDR1}{HDR2}
% produces two-column table with column headers HDR1 and HDR2.
\newenvironment{longlibtab2}[5]
{
 \begin{LongTable}{#1}{#2}{#3}
 \\ \topline
 \lhdr{#4}	&	\rhdr{#5}	\\ \capsep
 \endfirsthead
 \continuedcaption\\
 \topline
 \lhdr{#4}	&	\rhdr{#5}	\\ \capsep
 \endhead
}
{
  \end{LongTable}
}

% usage: \begin{LibEffTab}{TITLE}{XREF}{HDR2}{WD2}
% produces a two-column table with left column header "Element"
% and right column header HDR2, right column word-wrapped with
% width specified by WD2.
\newenvironment{LibEffTab}[4]
{
 \begin{libtab2}{#1}{#2}{lp{#4}}{Element}{#3}
}
{
 \end{libtab2}
}

% Same as LibEffTab except that it uses a long table.
\newenvironment{longLibEffTab}[4]
{
 \begin{longlibtab2}{#1}{#2}{lp{#4}}{Element}{#3}
}
{
 \end{longlibtab2}
}

% usage: \begin{libefftab}{TITLE}{XREF}
% produces a two-column effects table with right column
% header "Effect(s) if set", width 4.5 in. Used in "fmtflags effects"
% table in standard.
\newenvironment{libefftab}[2]
{
 \begin{LibEffTab}{#1}{#2}{Effect(s) if set}{4.5in}
}
{
 \end{LibEffTab}
}

% Same as libefftab except that it uses a long table.
\newenvironment{longlibefftab}[2]
{
 \begin{longLibEffTab}{#1}{#2}{Effect(s) if set}{4.5in}
}
{
 \end{longLibEffTab}
}

% usage: \begin{libefftabmean}{TITLE}{XREF}
% produces a two-column effects table with right column
% header "Meaning", width 4.5 in. Used in "seekdir effects"
% table in standard.
\newenvironment{libefftabmean}[2]
{
 \begin{LibEffTab}{#1}{#2}{Meaning}{4.5in}
}
{
 \end{LibEffTab}
}

% usage: \begin{libefftabvalue}{TITLE}{XREF}
% produces a two-column effects table with right column
% header "Value", width 3 in. Used in "basic_ios::init() effects"
% table in standard.
\newenvironment{libefftabvalue}[2]
{
 \begin{LibEffTab}{#1}{#2}{Value}{3in}
}
{
 \end{LibEffTab}
}

% usage: \begin{libefftabvaluenarrow}{TITLE}{XREF}
% produces a two-column effects table with right column
% header "Value", width 1.5 in. Used in basic_string_view effects
% tables in standard.
\newenvironment{libefftabvaluenarrow}[2]
{
 \begin{LibEffTab}{#1}{#2}{Value}{1.5in}
}
{
 \end{LibEffTab}
}

% Same as libefftabvalue except that it uses a long table and a
% slightly wider column.
\newenvironment{longlibefftabvalue}[2]
{
 \begin{longLibEffTab}{#1}{#2}{Value}{3.5in}
}
{
 \end{longLibEffTab}
}

% usage: \begin{liberrtab}{TITLE}{XREF} produces a two-column table
% with left column header ``Value'' and right header "Error
% condition", width 4.5 in. Used in regex Clause in the TR.

\newenvironment{liberrtab}[2]
{
 \begin{libtab2}{#1}{#2}{lp{4.5in}}{Value}{Error condition}
}
{
 \end{libtab2}
}

% Like liberrtab except that it uses a long table.
\newenvironment{longliberrtab}[2]
{
 \begin{longlibtab2}{#1}{#2}{lp{4.5in}}{Value}{Error condition}
}
{
 \end{longlibtab2}
}

% usage: \begin{lib2dtab2base}{TITLE}{XREF}{HDR1}{HDR2}{WID1}{WID2}{WID3}
% produces a table with one heading column followed by 2 data columns.
% used for 2D requirements tables, such as optional::operator= effects
% tables.
\newenvironment{lib2dtab2base}[7]
{
 %% no lines in the top-left cell, and leave a gap around the headers
 %% FIXME: I tried to use hhline here, but it doesn't appear to support
 %% the join between the leftmost top header and the topmost left header,
 %% so we fake it with an empty row and column.
 \newcommand{\topline}{\cline{3-4}}
 \newcommand{\rowsep}{\cline{1-1}\cline{3-4}}
 \newcommand{\capsep}{
  \topline
  \multicolumn{4}{c}{}\\[-0.8\normalbaselineskip]
  \rowsep
 }
 \newcommand{\bottomline}{\rowsep}
 \newcommand{\hdstyle}[1]{\textbf{##1}}
 \newcommand{\rowhdr}[1]{\hdstyle{##1}&}
 \newcommand{\colhdr}[1]{\multicolumn{1}{|>{\centering}m{#6}|}{\hdstyle{##1}}}
 %% FIXME: figure out a way to reuse floattable here
 \begin{table}[htbp]
 \caption{\label{#2}#1}
 \begin{center}
 \begin{tabular}{|>{\centering}m{#5}|@{}p{0.2\normalbaselineskip}@{}|m{#6}|m{#7}|}
 %% table header
 \topline
 \multicolumn{1}{c}{}&&\colhdr{#3}&\colhdr{#4}\\
 \capsep
}
{
 \bottomline
 \end{tabular}
 \end{center}
 \end{table}
}

\newenvironment{lib2dtab2}[4]{
 \begin{lib2dtab2base}{#1}{#2}{#3}{#4}{1.2in}{1.8in}{1.8in}
}{
 \end{lib2dtab2base}
}
