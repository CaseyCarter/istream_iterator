\begingroup
\def\hd{\begin{tabular}{lll}
          \textbf{Document Number:} & {\larger\docno}                 \\
          \textbf{Date:}            & \reldate                        \\
          \textbf{Audience:}        & Library Evolution Working Group \\
          \textbf{Author:}          & Casey Carter                    \\
          \textbf{Reply to:}        & casey@carter.net
          \end{tabular}
}
\newlength{\hdwidth}
\settowidth{\hdwidth}{\hd}
\hfill\begin{minipage}{\hdwidth}\hd\end{minipage}
\endgroup

\vspace{0.5cm}
\begin{center}
\textbf{\Huge \doctitle}
\end{center}

\vspace{0.5cm}

\chapter{Abstract}

Usage of \tcode{using namespace std;} means we break existing code in every new
\Cpp{} standard. Our mechanisms for name collision avoidance reduce but do not
eliminate the potential for collisions, do so by reducing the quality of the
Standard Library, and are not sustainable. Instead of demanding that users pay
the cost of avoiding name collisions by typing \tcode{std::} everywhere we
should provide an alternative mechanism that is equally convenient. This paper
proposes such a mechanism in a familiar guise:
\begin{codeblock}
using namespace std::cpp17;
\end{codeblock}

\begin{libtab2}{Tony Table}{tab:tony}{l|l}{Before}{After} %% FIXME
\begin{codeblock}
using namespace std;
using span = int[42];
vector<int> vec;
// {\color{remclr} ill-formed; ambiguous:}
span x;
\end{codeblock} & \begin{codeblock}
using namespace std::cpp17;
using span = int[42];
vector<int> vec;
// Ok:
span x;
\end{codeblock} \\ \rowsep

\begin{codeblock}
#include <range/v3/core.hpp>
using namespace std;
// {\color{remclr} ill-formed; ambiguous:}
auto first = ranges::begin(vec);
\end{codeblock} & \begin{codeblock}
#include <range/v3/core.hpp>
using namespace std::cpp17;
// Ok:
auto first = ranges::begin(vec);
\end{codeblock} \\
\end{libtab2}

\section{Revision History}
\subsection{Revision 1}
\begin{itemize}
\item Indicate that the shadow of an inline namespace is also an inline
  namespace.
\item Clarify that only entities that both are defined in \tcode{std} and were
  defined in \tcode{std} in an older language version are shadowed.
\item Add detailed example.
\end{itemize}
\subsection{Revision 0}
\begin{itemize}
\item Sprang fully formed from the forehead of Zeus.
\end{itemize}

\chapter{Problem description}

Name collisions between the Standard Library and user programs that pull the
entire standard library into their namespaces - or the global namespace - via
\tcode{using namespace std;} are a perennial problem for \Cpp. WG21 is divided
as to whether we should be responsible for avoiding such collisions. On the one
hand no one wants to break existing code, and on the other no one wants to
pessimize name choices in the library. This is a constant source of problems for
us, with the typical result being that big companies with loud voices in the
committee are insulated from name collisions and Joe Programmer is left out in
the cold.

We use two mechanisms to avoid collisions: either replace the good
name\footnote{Perhaps ``chosen name'' would be a better term than ``good name''
to describe the output of LEWG.} with a bad name - which is problematic for
obvious reasons - or relocate the good name into a subnamespace of \tcode{std} -
which is problematic for less obvious reasons. Subnamespaces make it harder for
people who do fully qualify names to use the standard library - they're paying
for what they don't use.

Neither mechanism avoids collions entirely: the replacement name may be less
likely to cause collisions, but the potential is still there. Neither mechanism
is sustainable: as the body of \Cpp{} code grows over time, and more names are
added to the standard library, the supply of usable names gradually dwindles.
We'll eventually run out of names to use directly in \tcode{std} either as
replacements or subnamespace names, and have to start increasing the nesting
depth of subnamespaces rendering the library unusable without namespace
gymnastics.

For a recent example, consider P0631R4 ``Math Constants''~\cite{P0631R4} which
adds definitions of common mathematical constants to the standard library like
\tcode{pi} and \tcode{e}. The consensus opinion on the reflector was to move the
constants into \tcode{std::math} to avoid collisions. Note that \tcode{math} is
far from an uncommon name; we've traded one problem for another.

There have even been rumbles in LEWG about deprecating
\tcode{using namespace std;} to \emph{force} users to fully qualify names they
use from the standard library or write using declarations for individual names.
It's hard to envision this kind of adversarial tactic being well-received by
users. What if, instead of harming ease of use by trying to get users to stop
doing this thing we want them to not do, we gave them an alternative that
doesn't require them to do more work?


\chapter{Proposal}
If users won't write using declarations for the standard library names they
want to use unqualified, why don't we do it for them? This paper proposes adding
nested namespaces \tcode{std::cpp17} and \tcode{std::cpp20}. Within the
nested namespace \tcode{std::cpp$X$} there will be a shadow of the entire name
structure of the standard library as it was defined in \Cpp{} version $X$.

To be precise, in each standard library header:
\begin{itemize}
\item For each namespace \tcode{std::$\cdots$::$N$} defined in that header
  \added{that was also defined} in \Cpp$X$, there shall be a corresponding shadow namespace
  \tcode{std::cpp$X$::$\cdots$::$N$}\added{; the shadow namespace shall be an inline
  namespace if and only if the namespace being shadowed is an inline namespace}.

\item For each namespace alias \tcode{std::$\cdots$::$A$} defined in that
  header \added{that was also defined} in \Cpp$X$ that denotes a namespace
  \tcode{std::$\cdots$::$N$}, there shall be
  a corresponding shadow namespace alias
  \tcode{std::cpp$X$::$\cdots$::$A$} that denotes namespace
  \tcode{std::cpp$X$::$\cdots$::$N$}.

\item Each other name $E$ in namespace \tcode{std::$\cdots$::$N$}
  defined in that header \added{that was also defined} in \Cpp$X$ shall have a corresponding
  \grammarterm{using-declaration} in \tcode{std::cpp$X$::$\cdots$::$N$}:
  \begin{codeblock}
  using ::std::@$\cdots$@::@$N$@::@$E$@;
  \end{codeblock}

\item Each enumerator $E$ of unscoped enumeration type $T$ defined in namespace
  \tcode{std::$\cdots$::$N$} in that header \added{that was also defined} in \Cpp$X$ shall have
  a corresponding \grammarterm{using-declaration}
  in \tcode{std::cpp$X$::$\cdots$::$N$}:
  \begin{codeblock}
  using ::std::@$\cdots$@::@$N$@::@$E$@;
  \end{codeblock}
\end{itemize}
The end result is that \tcode{std::cpp$X$} in each header mirrors
the structure and content that existed in that header in \tcode{std} in \Cpp$X$
\added{, except for any entities that have been removed from the library since \Cpp$X$}.
The meaning of those names may change in future revisions of \Cpp, but the set
of visible names will not.

\newpage
\section{Example}

Were the header \tcode{<foo>} defined in the working draft:

\begin{codeblock}
namespace std {
  template</**/> struct new_fancy_vector; // Added in C++20.

  enum some_enum { X, Y }; // Added in C++20

  namespace chrono {
    using nanoseconds = /**/;

    inline namespace chrono_literals {
      constexpr chrono::nanoseconds operator""ns(unsigned long long);
    }
  }

  // We would shadow all of its contents in cpp20 as:
  namespace cpp20 {
    using std::new_fancy_vector, std::some_enum, std::X, std::Y;

    namespace chrono {
      using std::chrono::nanoseconds;

      inline namespace chrono_literals {
        using std::chrono::chrono_literals::operator""ns;
      }
    }
  }

  // And a subset of the contents in cpp17 as:
  namespace cpp17 {
    namespace chrono {
      using std::chrono::nanoseconds;

      inline namespace chrono_literals {
        using std::chrono::chrono_literals::operator""ns;
      }
    }
  }
}
\end{codeblock}

This well-formed C++17 program:
\begin{codeblock}
#include <foo>
#include <vector>

using namespace std;

struct fancy_new_vector { /**/ };
int main() {
  fancy_new_vector v = std::vector<int>{1,2,3};
}
\end{codeblock}

would break when compiled with a C++20 implementation. Replacing
\tcode{using namespace std;} with \tcode{using namespace std::cpp17;} will allow
the program to continue to compile. Had we shadow namespaces in C++17,
the programmer could have written \tcode{using namespace std::cpp17;} originally,
and would have had no issues switching to a C++20 implementation, allowing them
to deal with such name collisions at leisure upon updating this translation unit
(TU) from \tcode{cpp17} to \tcode{cpp20}.

\section{Discussion}

How can this prevent name collisions for programs that
\tcode{using namespace std;}? Simple - it can't. It can, however, prevent name
collisions from names newly added in \Cpp20 for programs that
\tcode{using namespace std::cpp17;}, however, and prevent name collisions
from names newly added in \Cpp23 for programs that
\tcode{using namespace std::cpp20;}. These shadow namespaces are exactly the
mechanism required. Replacing \tcode{using namespace std;} with
\tcode{using namespace std::cpp17;} requires almost no additional effort from
users - it's 7 more keytrokes per TU, rather than 5 keystrokes per reference to
the standard library - and the replacement is easily toolable - a \tcode{sed}
script could update an entire source tree in no time with almost no chance for
error.

The simultaneous presence of multiple shadow namespaces has the additional
benefit of allowing for gradual transition of a codebase from an old standard
library to a new one. If you replace \tcode{using namespace std;} with
\tcode{using namespace std::cpp17;}, your program is substantially more likely
to compile after throwing the \Cpp20 compiler switch. You are then free to
add references to / \grammarterm{using-declarations} for fully-qualified names
in \tcode{std} to get at new features without touching the rest of the TU,
change 17 to 20 TU-by-TU, or even stay at \tcode{cpp17} indefinitely while
continuing to upgrade compilers and core language versions in the expectation
that your standard library implementation will continue to support
\tcode{std::cpp17}.

\section{What about Argument Dependent Lookup?}
Argument Dependent Lookup\cxxiref{basic.lookup.argdep} pokes some holes in the
proposal. For example, this \Cpp20 program:
\begin{codeblock}
#include <vector>
using namespace std::cpp20;
int frobnozzle(vector<int>&) { /**/ }
int main() {
  vector<int> vec;
  frobnozzle(vec);
}
\end{codeblock}
would be broken if we add a function with the signature
\tcode{std::frobnozzle(vector<int>\&)} in \Cpp23. Since
\tcode{std::cpp20::vector} and \tcode{std::vector} are the same template, ADL
for the unqualified call to \tcode{frobnozzle} in \tcode{main} will find the
new signature in \tcode{std} - shadow namespaces or not - resulting in overload
resolution ambiguity. This proposal greatly reduces the surface for name
collisions, but doesn't eliminate them completely when ADL is a factor.

There is ongoing work in both EWG and LWG to address issues with ADL which will
likely help with this problem if not solve it outright.

\section{Messaging}
Together with making this change, we as a community need to change our messaging
on \grammarterm{using-directive}s. The Standard should provide normative
encouragement for implementations to
\href{https://i.imgflip.com/2n1mwr.jpg}{diagnose \tcode{using namespace std;}
as a dangerous practice and suggest \tcode{using namespace std::cpp$X$}
instead.} WG21 should change its guidance from ``never write \tcode{using
namespace std}'' and/or ``write \tcode{using namespace std;} in CPP files'' to
``write \tcode{using namespace std::cpp$X$} in CPP files if you like, and never
ever write \tcode{using namespace std;} or we will gleefully break your program.''

\section{Brains...}
We should simultaneously zombify all names matching the regular expression
\tcode{cpp$\backslash$d+} for past and future standardization to defend against
macroization. This both prevents defining \tcode{cpp23} in \Cpp20 programs
which would then fail to compile in \Cpp23, and allows vendors to indefinitely
maintain support for older versions of the standard in e.g. \tcode{std::cpp11}
or \tcode{std::cpp98}.\footnote{At least until 2098 when WG21 must decide
whether to drop support for \Cpp98 or define the new shadow namespace
as \tcode{std::cpp2098}.}

\section{How many versions?}
How far back should the required shadow namespaces go? The proposal above
obviously suggests at least coverage of ``current'' and ``previous'' Standard
revisions. Should we go back further? Should we start at two and grow to $N$?
Grow indefinitely? This seems to be a question of (a) the cost to
implementations to maintain (or implement, for fresh implementations) old shadow
namespaces and (b) the cost to WG21 to maintain the specifications of such in
the IS.

\section{IS Maintenance}
WG21 maintenance costs depend on the form of wording for this feature that's
acceptable to LWG and the Editor. Wording could be as simple as listing the
above bullets that describe the proposal in the library front matter and
indicating the number of required shadow namespaces,
which would require no maintenance,
or as detailed as realizing the complete set of required declarations concretely
in each header. The latter would require per-IS-cycle maintenaince in which we
strike the oldest shadow namespace from each header synopsis - if we choose not
to maintain them indefinitely - and duplicate the most recent shadow namespace
into the shadow namespace for the current standard. It would also require
careful monitoring by LWG to ensure that each proposal that adds a new name to
\tcode{std} simultaneously adds that same name to the current shadow namespace.

\section{Painting the bikeshed}
Obviously \tcode{std::cpp$X$} is not the only possible naming scheme for the
shadow namespaces. \tcode{std$X$} has already been suggested to the author,
which has two benefits:
\begin{itemize}
\item those namespaces have been reserved since \Cpp17, and
\item \tcode{using namespace std17;} is 5 keystrokes shorter than
  \tcode{using namespace std::cpp17;}
\end{itemize}
although it seems a shame to paint ourselves into a corner and kill the dream
of \tcode{std2} die completely.


\chapter{Technical specifications}
Wording for this proposal will be straightforward - but possibly voluminous -
once LWG and the Editor decide what form the wording should take. Rather than
providing page upon page of wording for alternative forms here which would only
contribute to the word cloud for the pre-meeting mailing, the author awaits
direction on the desired form.

%% Feedback:
%% Billy: Concerns that the mirroring increases the number of symbols the
%% compiler must parse by a factor of three. (There's a cost for this feature
%% that needs to be considered.)
